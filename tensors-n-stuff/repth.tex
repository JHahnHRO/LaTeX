% !TeX root = tensors.tex
% !TeX spellcheck = en_GB

\begin{definition}
Let $G$ be a group. A \udot{representation} of $G$ is a vector space $W$ together with a family of linear maps $\phi_g: W\to W$ such that
\[\forall g,h\in G: \phi_g \circ \phi_h = \phi_{gh}\]
holds. With these maps we can view the elements of $G$ as linear operators acting on $V$ and if we want to simplify notation we will simply write this action as multiplication, i.e. we will write $gw$ instead of $\phi_g(w)$ for group elements $g\in G$ and vectors $w\in W$.
\end{definition}

\begin{example}
The vector space $V$ is a representation for the group $G=O(V)$.
\end{example}

\begin{example}[Trivial representations]
Any vector space can be upgraded to a representation for any group, the so called \udot{trivial representation}, by defining $\phi_g$ to be the identity map $\id_V$ for all $g\in G$.
\end{example}

\begin{theorem}[New representations from old]
If $(W,\phi_g)$, $(W',\phi'_g)$ are representations of $G$, then the following vector spaces are also representations w.r.t. the given actions of $G$:
\begin{enumerate}
\item any subspace $U\leq W$ with $\forall g\in G: \phi_g(U) \subseteq U$ (a so called \udot{invariant subspace}) with $gu := \phi_g(u)$,
\item any quotient $W/U$ by an invariant subspace with $g(w+U) := \phi_g(w)+U$
\item the direct product $W_1\times W_2$ with $g(w_1,w_2) := (\phi_g(w_1),\phi_g'(w_2))$, and
\item the tensor product $W_1 \otimes W_2$ with $g(w_1\otimes w_2) := \phi_g(w_1)\otimes\phi_g(w_2)$.
\end{enumerate}
\end{theorem}

\begin{example}[Trivial sub-representations]
Every representation has two obvious subspaces that are invariant, namely the whole space and the subspace $\set{0}$.
\end{example}

\begin{definition}[Irreducible representations]
Let $W\neq\set{0}$ be a representation of $G$. If the two obvious invariant subspaces $W$ are the only invariant subspaces, then $W$ is said to be an \udot{irreducible representation}.
\end{definition}

\begin{example}[One-dimensional representations]
Every one-dimensional representation is irreducible for dimension reasons.
\end{example}

\begin{example}
The euclidean space $V$ is irreducible as a representation for $O(V)$. To see this, assume that $U\leq V$ is any non-zero, invariant subspace. Let $u\in U\setminus\set{0}$.
\end{example}

\begin{theorem}[Schur's lemma]

\end{theorem}