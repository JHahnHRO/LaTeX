% !TeX root = kristallographie_skript.tex
% !TeX spellcheck = de_DE
\documentclass[fontsize=11pt,fleqn,a4paper]{scrartcl}
\input{_preamble/language_en.tex}
\input{_preamble/math_general.tex}
%%% Own symbols and operators
\newcommand{\IN}{\mathbb{N}}
\newcommand{\IZ}{\mathbb{Z}}
\newcommand{\IQ}{\mathbb{Q}}
\newcommand{\IR}{\mathbb{R}}
\newcommand{\IC}{\mathbb{C}}
\newcommand{\IK}{\mathbb{K}}
\newcommand{\IF}{\mathbb{F}}

\DeclarePairedDelimiter{\abs}{\lvert}{\rvert}
\DeclarePairedDelimiter{\norm}{\lVert}{\rVert}
\DeclarePairedDelimiter{\ceil}{\lceil}{\rceil}
\DeclarePairedDelimiter{\floor}{\lfloor}{\rfloor}

\newcommand{\isomorphic}{\cong}
\newcommand{\homotopic}{\simeq}

\renewcommand{\Im}{\operatorname{\mathfrak{Im}}}
\renewcommand{\Re}{\operatorname{\mathfrak{Re}}}

\DeclareMathOperator{\colim}{colim}

\DeclareMathOperator{\id}{id}
\DeclareMathOperator{\Hom}{Hom}
\DeclareMathOperator{\End}{End}
\DeclareMathOperator{\Irr}{Irr}
\DeclareMathOperator{\IBr}{IBr}

\DeclareMathOperator{\Ind}{Ind}
\DeclareMathOperator{\Res}{Res}

\DeclareMathOperator{\tr}{tr}
\DeclareMathOperator{\sgn}{sgn}
\DeclareMathOperator{\diag}{diag}
\DeclareMathOperator{\ord}{ord}

\DeclareMathOperator{\CharFld}{char}
\DeclareMathOperator{\QuotFld}{Quot}

\DeclareMathOperator{\im}{im}
\DeclareMathOperator{\rad}{rad}

\input{_preamble/math_theorems.tex}
\input{_preamble/math_styles.tex}
\input{_preamble/tikz.tex}
%%%% Styles for algorithms

\usepackage{listings}
\lstset{%
	basicstyle = \ttfamily\small,
	tabsize = 3
}
% Use theorem environment for algorithm descriptions
\newtheoremstyle{algorithms} % Name
			{\bigskipamount}    % Space above
			{\bigskipamount}    % Space below
			{\nopagebreak}      % Body font, also suppress pagebreak between "Theorem 3.14:" and text
			{}                  % Indent amount
			{\bfseries}         % Theorem head font
			{:}                 % Punctuation after theorem head
			{\newline}          % Space after theorem head
			{}                  % Theorem head spec (can be left empty, meaning 'normal')
\theoremstyle{algorithms}
\swapnumbers
\newtheorem{algorithm}{Algorithmus}
% Change numbering of algorithms to include chapter
\renewcommand{\thealgorithm}{A\arabic{algorithm}}
%%%% Tables and figures
\numberwithin{table}{section}
\numberwithin{figure}{section}



\input{_preamble/text.tex}
%\input{_preamble/bibtex_only.tex}
%\input{_preamble/biblatex_bibtex.tex}
%\input{_preamble/biblatex_biber.tex}

% !! Hyperref before imakeidx !!
\input{_preamble/hyperref.tex}
%\input{_preamble/indicies.tex}



\author{Johannes Hahn}
\title{TITLE}
%\subtitle{}

\hypersetup{
pdfinfo=
	{  
		Title={TITLE},
		Author={Johannes Hahn},
		Keywords={KEYWORDS},
		Subject={SUBJECT}
	}
}


\title{Kristallographie}
\author{Johannes Hahn \and Andrea Hanke}

\begin{document}

\maketitle

\section{Gruppentheorie}

\section{Kristalle}

\begin{definition}[Kristalle]
Ein \udot{Kristall} (auch \udot{Kristallgitter}) ist eine Punktmenge $\Lambda\subseteq\IR^3$ (gedacht als die Menge aller Atome im Kristall), die ...
\begin{enumerate}
\item ... Translationssymmetrie hat, d.h. es gibt Vektoren $t_1,t_2,t_3\in\IR^3$ in drei unabhängige Richtungen, sodass immer, wenn $x\in\Lambda$ ein Punkt im Kristall ist, $x+k_1 t_1+k_2t_2+k_3t_3$ auch ein Punkt im Kristall ist für alle ganzen Zahlen $k_1,k_2,k_3\in\IZ$.
\item ... aus isolierten Punkten besteht, d.h. es gibt einen Mindestabstand $\delta>0$, sodass sich keine zwei Punkte $x,y\in\Lambda$ näher als $\delta$ kommen: $x\neq y \implies\norm{x-y}\geq\delta$.
\end{enumerate}
\end{definition}

\begin{remark}
Insbesondere bedeutet dass, dass es nur abzählbar viele Punkte im Gitter gibt.
\end{remark}

\begin{definition}
Die \udot{Symmetriegruppe} eines Kristallgitters $\Lambda$ ist die Gruppe aller starren Bewegungen (=Isometrien), die das Gitter in sich selbst abbilden:
\[\Aut(\Lambda) := \Set{s\in \operatorname{Isom}(\IR^3) | s(\Lambda)=\Lambda}\]
Nach Definition enthält $\Aut(\Lambda)$ mindestens die drei Translationen $x\mapsto x+t_i$. Die Menge \emph{aller} Translationen, die $\Lambda$ in sich selbst abbilden, sind eine Untergruppe von $\Aut(\Lambda)$.
\end{definition}

\begin{remark}
Weil die Punkte in $\Lambda$ einen Mindestabstand haben, ist die Translationsuntergruppe diskret, d.h. sie enthält eine Basis: Drei Translationen $\tau_1,\tau_2,\tau_3\in\Aut(\Lambda)$, sodass sich jede beliebige Translation $\tau\in\Aut(\Lambda)$ auf eindeutige Weise als $\tau_1^{k_1}\circ\tau_2^{k_2} \circ \tau_3^{k_3}$ mit $k_1,k_2,k_3\in\IZ$ schreiben lässt.

Die Translationsuntergruppe ist also zur Gruppe $(\IZ^3,+)$ isomorph.
\end{remark}

\begin{example}
Umgekehrt: Wenn $v_1,v_2,v_3\in\IR^3$ drei beliebige, linear unabhängige Vektoren sind, dann ist $\Lambda:=\IZ v_1+\IZ v_2 + \IZ v_3 = \Set{k_1 v_1+k_2 v_2+k_3 v_3 | k_1,k_2,k_3\in\IZ}$ ein Gitter, dessen Translationsuntergruppe die drei Translationen $\tau_i:=x\mapsto x+v_i$ als (eine mögliche von vielen) Basis hat.
\end{example}

\begin{definition}
Es sei $\Lambda$ ein Kristallgitter und $T$ die Gruppe aller Translationen, die $\Lambda$ invariant lassen.

Eine \udot{Basiszelle von $\Lambda$} ist ein Paar $(Z,A)$ bestehend aus einem (konvexer, kompakter) Polyeder $Z\subseteq\IR^3$ und einer Punktmenge $A\subseteq Z$, sodass
\begin{enumerate}
\item ... die Translate von $A$ ganz $\Lambda$ überdecken, d.h. $\Lambda=\bigcup_{t\in T} tA$.
\item ... die Translate von $Z$ ganz $\IR^3$ überdecken, d.h. $\IR^3 = \bigcup_{t\in T} tZ$.
\item ... die Translate von $Z$ im wesentlichen disjunkt sind, d.h. $Z \cap tZ$ ist leer oder höchstens eine Seitenfläche von $Z$, wenn $t\in T\setminus\set{1}$ ist.
\end{enumerate}

Eine Basiszelle des kleinstmöglichen Volumens heißt \udot{elementare Basiszelle} des Gitters.
\end{definition}

\begin{remark}
Da $Z$ kompakt ist und $\Lambda$ diskret, muss $A=Z\cap\Lambda$ endlich sein.
\end{remark}

\end{document}