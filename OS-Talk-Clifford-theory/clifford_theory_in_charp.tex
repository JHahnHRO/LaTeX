% !TeX root = repth.tex
% !TeX spellcheck = en_GB

\begin{theorem}[Green's indecomposability theorem]
Let $G$ be a group and $H\unlhd\unlhd G$ a subnormal group with $\abs{G:H}$ a power of $p$. If $\IF$ is perfect and a splitting field for $G$ and its subgroups, then $\Ind_H^G$ maps indecomposable $\IF H$- to indecomposable $\IF G$-modules.
\end{theorem}
\begin{proof}
By induction we assume $H\unlhd G$, $\abs{G:H}=p$. Then let $M\in \IF H\mathsf{-mod}$ be indecomposable and $T:=T_G(M)$ its inertial group. Because $H\leq T\leq G$, we have to consider two cases:

\medbreak
Case 1: $H=T$. In this case, there are exactly $p$ conjugates of $M$. If $\Ind_H^G(M) = V_1\oplus V_2$, then
\[\Res_H^G(V_1)\oplus\Res_H^G(V_2) = \Res_H^G\Ind_H^G(M) = \bigoplus_{xH} {^x M}\]
and $\Res_H^G(V_i)$ contains a full $G$-orbit of indecomposable submodules. This means that $V_i$ must already be all of the RHS so that $\Ind_H^G(M)$ is indecomposable.

\medbreak
Case 2: $T=G$. In this case, $M$ is $G$-invariant. Therefore there are isomorphisms $\alpha_g: M\to M$ with $\alpha_g(hm) = ghg^{-1} \alpha_g(m)$.

$E:=\End_{\IF G}(\Ind_H^G(M))$ is a $G/H$-graded algebra with
\[E_g:=\Set{f | f(1\otimes M)\subseteq g^{-1}\otimes M}\]
Then $f_g := x\otimes m\mapsto xg^{-1}\otimes\alpha_g(m)$ is a well-defined, $G$-linear and bijection mapping $1\otimes M$ to $g^{-1}\otimes M$. Thus $E$ is a crossed $G/H$-graded algebra.

Now observe that $E_1=\End_{\IF H}(M)$. It is clear, that every $G$-linear $f$ with $f(1\otimes M)\subseteq 1\otimes M$ induces a $H$-linear map $M\to M$. If conversely $f:M\to M$ is $\IF H$-linear, then $\Ind_H^G(f):g\otimes m\mapsto g\otimes f(m)$ is a $\IF G$-linear map that restricts to $f$.

By induction assumption $E_1=\End_{\IF H}(M)$ is local because $M$ is indecomposable so that $E$ must be local as well by \ref{g_graded_algebras:Green_theorem_on_local_algebras}. In particular $\Ind_H^G(M)$ is indecomposable.
\end{proof}

\begin{theorem}
Let $M$ be an indecomposable $\IF G$-module with vertex $V$. If $P\in Syl_p(G)$ is such that $V\leq P$, then $\abs{P:V} \mid \dim M$. In particular $\abs{P:D} \mid \dim(M)$ if $D$ is a defect group of the block of $V$.
\end{theorem}
\begin{proof}
Choose a source $S \mid \Res_V^G(M)$. Then $M \mid \Ind_V^G(S)$ so that 
\[\Res_P^G(M) \mid \bigoplus_{PxV} \Ind_{P\cap{^x V}}^P \Res_{P\cap{^x v}}^{^x V}(^x S)\]
If we decompose all $\Res_{P\cap{^x V}}^{^x V}(^x S)$ into indecomposables $Y_{x,i}$, we get a decomposition of $\Res_P^G(M)$ as sum of indecomposables (by Green's theorem) of the form $\Ind_{P\cap{^x V}}^P(Y_{x,i})$. In particular, each of these indecomposables has dimension divisible by $\abs{P:P\cap{^x V}}$. Since $P\cap{^x V} \leq {^x V}$, this index is a multiple of $\frac{\abs{P}}{\abs{^x V}}=\abs{P:V}$.
\end{proof}

\begin{definition}[Height of a module]
Let $M$ be an irreducible $\IF G$-module in some block $B\in Bl(\IF[G])$, $D=Def(B)$ and $P\in Syl_p(G)$ above $D$. Then $\abs{P:D} \mid \dim_\IF(M)$ because $vx(M)\leq_G D$. The natural number $ht(M)$ with
\[p^{ht(M)} = \left(\frac{\dim_\IF(M)}{\abs{P:D}}\right)_p\]
is called the \udot{height} of the module.
\end{definition}

\begin{corollary}
All one-dimensional modules have height zero and lie in blocks with $def(B)=Syl_p(G)$.
\end{corollary}

\begin{remark}
Height zero modules are somewhat analogous to linear characters, except that they exist in all blocks.
\end{remark}

\begin{theorem}[Zeros of ordinary characters] \label{vertices:zero_of_characters_outside_of_vertex}
If $M\in\mathcal{O}[G]$ is an indecomposable lattice with character $\chi$ and vertex $V$, then
\[g_p \notin \bigcup_g {^gV} \implies \chi(g) = 0\]
where $g_p$ denotes the $p$-part of $g$. In particular $\chi_{G\setminus G_{p'}} = 0$ if $M$ is projective.
\end{theorem}
\begin{proof}
Set $g_p =: u$,$g_{p'}=: s$, and $P:=\langle u\rangle$ and consider $\Res_P^G(M)$. By \ref{vertex_source:restriction_to_subgroups}, all indecomposable summands $N$ of $\Res_P^G(M)$ have vertex contained in some $P\cap{^g V}$. By assumption none of these groups contain $u$ so that they are all contained in the unique maximal subgroup $\langle u^p \rangle$ of $P$.

Inducing from subgroups of $P$ up to $P$ preserves indecomposable modules by Green's indecomposability theorem. Therefore $N$ (which is a summand of such an induced module) is itself equal to an induced module $\Ind_{\langle u^p\rangle}^{\langle u\rangle}(N_0)=\bigoplus_{k=0}^{p-1} u^k \otimes N_0$. In particular, $u$ acts as a $p$-cycle on $\set{N_0, u\otimes N_0, u^2\otimes N_0, \ldots, u^{p-1}\otimes N_0}$ and thus has eigenvalues $1,\zeta_p, \zeta_p^2, \ldots, \zeta_p^{p-1}$ each with multiplicity $\dim N_0$. In particular, $u$ acts with trace zero on $N$.

Now consider the action of $s$ on $N$. Since $s$ is $p$-regular, the $\langle s\rangle$ acts semisimple so that we can decompose $N$ into eigenspaces of $s$. Since $u$ and $s$ commute, all eigenspaces of $s$ are $u$-invariant. Since $N$ is indecomposable, this means that $s$ acts diagonally already. But this means $\tr(us \curvearrowright N)$ is a multiple of $\tr(u \curvearrowright N)=0$ and therefore zero as well.

This holds for all indecomposable summands $N \mid \Res_P^G(M)$ and therefore for $\Res_P^G(M)$ as well.
\end{proof}