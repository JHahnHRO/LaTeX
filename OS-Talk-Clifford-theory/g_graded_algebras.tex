% !TeX root = clifford.tex
% !TeX spellcheck = en_GB

\begin{definition}[Graded algebras]
Let $G$ be a group. A \emph{$G$-graded $k$-algebra} is a $k$-algebra $A$ endowed with a decomposition $A=\bigoplus_{g\in G} A_g$ into $k$-submodules such that $1\in A_1$ and $\forall g,h: A_g A_h\subseteq A_{gh}$.

\medbreak
$A$ is a \emph{crossed} $G$-graded algebra if each $A_g$ contains a unit.
\end{definition}

\begin{remark}
The homogeneous components of a crossed $G$-graded algebra all have the same dimension because $A_g = u_g A_1 = A_1 u_g$ if $u_g\in A_g\cap A^\times$. In particular: The units $u_g$ are unique up to multiplication with a unit from $A_1^\times$.
\end{remark}

\begin{example}[Group algebras and twisted group algebras]
$k[G]$ is a crossed $G$-graded algebra. More generally $k_\alpha[G]$ is a crossed $G$-graded algebra for all 2-cocycles $\alpha\in Z^2(G,k^\times)$.
\end{example}

\begin{example}[Quotient groups]
If $A$ is $\Gamma$-graded and $\phi:\Gamma\to G$ is a group homomorphism, then $A$ is also $G$-graded via
\[A_g := \bigoplus_{\gamma\in\phi^{-1}(g)} A_\gamma\]

If $A$ is crossed as a $\Gamma$-graded algebra and $\phi$ surjective, then it is also crossed as a $G$-graded algebra.
\end{example}

\begin{example}[Subgroups]
If $A$ is a $G$-graded algebra and $X\subseteq G$ a subset, then define
\[A_X := \bigoplus_{g\in X} A_g\]

Note that $A_X A_Y \subseteq A_{XY}$ so that if $H\leq G$ is a subgroup, then is a $H$-graded subalgebra of $A$. If $A$ is fully graded/crossed, then $A_H$ is fully graded/crossed too.

Furthermore: If $X=gH$ is a coset, then $A_X$ is a $A_{^g H}$-$A_H$-bimodule.
\end{example}

\begin{example}[Tensorproducts of graded algebras]
\begin{itemize}
\item  Let $A$ and $B$ be two (crossed) $G$-graded $k$-algebras. Then $A\otimes_k B$ is a (crossed) $G\times G$-graded algebra via $(A\otimes B)_{g,h} = A_g\otimes B_h$.

\item The diagonal subalgebra of the tensor product
\[A \odot B := \sum_{g\in G} A_g\otimes B_g \subseteq A\otimes B\]
is a (crossed) $G$-graded algebra.

\item $\odot$ is associative and $k[G]$ is the neutral element.

\item Note that if $A=k_\alpha[G]$, $B=k_\beta[G]$ for $\alpha,\beta\in Z^2(G,k^\times)$, then $A\odot B \isomorphic k_{\alpha\beta}[G]$.

\item This shows that $H^2(G,k^\times)$ acts on the category of $G$-graded algebras via $A\mapsto k_\alpha[G] \odot A$.

\item Moreover: If $V\in A\mathsf{-Mod}, W\in B\mathsf{-Mod}$, then $V \otimes_k W$ is naturally a $A\odot B$-module.
\end{itemize}
\end{example}

\begin{proposition}[Conjugation action]
Let $A$ be a crossed $G$-graded algebra. Then

Conjugation with $u_g\in A_g\cap A^\times$ defines a homomorphism $G\to \operatorname{Out}(A_1)$ or more generally to $\operatorname{gradedAut}(A)/\operatorname{Inn}(A_1)$.

Therefore $G$ acts by conjugation on everything that is invariant under multiplication with $A_1$ in an appropriate sense, e.g. the center of $A_1$, the centraliser $C_A(A_1)$, the set of two sided ideals of $A_1$, the blocks of $A_1$, the module category of $A_1$, $\Hom_{A_1}(X,Y)$ for modules, ...
\end{proposition}
\begin{proof}
$u_g u_h$ is a unit in $A_g A_h\subseteq A_{gh}$ and therefore $u_g u_h=u_{gh} a$ for some $a\in A_1^\times$. Therefore $\kappa_{u_{gh}} \equiv \kappa_{u_g}\circ\kappa_{u_h} \mod Inn(A_1)$ where $\kappa_u$ is conjugation with $u$.
\end{proof}