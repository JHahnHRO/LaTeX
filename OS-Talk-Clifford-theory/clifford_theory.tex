% !TeX root = clifford.tex
% !TeX spellcheck = en_GB

\begin{convention}
Fix a finite group $G$ and a crossed $G$-graded $k$-algebra $A$ for $k$ some commutative ring. Assume that $k$ and $A$ are nice enough such that the Krull-Schmidt theorem holds whenever we want it to hold. That is satisfied if $k$ is a complete DVR and $A_1$ is finitely generated as a $k$-module for example.

In particular, the block decompositions actually exist.
\end{convention}

\begin{remark}
Remember that $G$ acts by conjugation on $C_A(A_1)$. In particular it acts on $Z(A_1)$ and also on $Bl(A_1)$.
\end{remark}

\begin{definition}
Let $b\in Bl(A_1)$ be a block. Define the \udot{inertial group} $I_G(b)$ as the stabiliser of $b$ w.r.t. the conjugation action.
\end{definition}

\begin{remark}
In the example of $A=k[\Gamma]$ with the $\Gamma/N$-grading, $A_1$ is just $k[N]$. The inertial group $I_\Gamma(b)$ is usually defined as the stabiliser of the conjugation action of $\Gamma$ on $k[N]$. Since $N$ acts trivially on $C_{k[\Gamma]}(k[N])$, this factors through $G=\Gamma/N$ and gives exactly the same conjugation action on $C_{k[\Gamma]}(k[N])$. Therefore $N \leq I_\Gamma(b)$ and we can view $I_\Gamma(b)$ as a subgroup of $G$ instead. Then we get $I_{G}(b)=I_\Gamma(b)/N$.

\medbreak
Moreover $C_\Gamma(N) \leq I_\Gamma(b)$ in that example.
\end{remark}
