% !TeX root = derived_functors.tex
% !TeX spellcheck = en_GB
\documentclass[fontsize=11pt,fleqn,a4paper]{scrartcl}
\input{_preamble/language_en.tex}
\input{_preamble/math_general.tex}
%%% Own symbols and operators
\newcommand{\IN}{\mathbb{N}}
\newcommand{\IZ}{\mathbb{Z}}
\newcommand{\IQ}{\mathbb{Q}}
\newcommand{\IR}{\mathbb{R}}
\newcommand{\IC}{\mathbb{C}}
\newcommand{\IK}{\mathbb{K}}
\newcommand{\IF}{\mathbb{F}}

\DeclarePairedDelimiter{\abs}{\lvert}{\rvert}
\DeclarePairedDelimiter{\norm}{\lVert}{\rVert}
\DeclarePairedDelimiter{\ceil}{\lceil}{\rceil}
\DeclarePairedDelimiter{\floor}{\lfloor}{\rfloor}

\newcommand{\isomorphic}{\cong}
\newcommand{\homotopic}{\simeq}

\renewcommand{\Im}{\operatorname{\mathfrak{Im}}}
\renewcommand{\Re}{\operatorname{\mathfrak{Re}}}

\DeclareMathOperator{\colim}{colim}

\DeclareMathOperator{\id}{id}
\DeclareMathOperator{\Hom}{Hom}
\DeclareMathOperator{\End}{End}
\DeclareMathOperator{\Irr}{Irr}
\DeclareMathOperator{\IBr}{IBr}

\DeclareMathOperator{\Ind}{Ind}
\DeclareMathOperator{\Res}{Res}

\DeclareMathOperator{\tr}{tr}
\DeclareMathOperator{\sgn}{sgn}
\DeclareMathOperator{\diag}{diag}
\DeclareMathOperator{\ord}{ord}

\DeclareMathOperator{\CharFld}{char}
\DeclareMathOperator{\QuotFld}{Quot}

\DeclareMathOperator{\im}{im}
\DeclareMathOperator{\rad}{rad}

\input{_preamble/math_theorems.tex}
\input{_preamble/math_styles.tex}
\input{_preamble/tikz.tex}
%%%% Styles for algorithms

\usepackage{listings}
\lstset{%
	basicstyle = \ttfamily\small,
	tabsize = 3
}
% Use theorem environment for algorithm descriptions
\newtheoremstyle{algorithms} % Name
			{\bigskipamount}    % Space above
			{\bigskipamount}    % Space below
			{\nopagebreak}      % Body font, also suppress pagebreak between "Theorem 3.14:" and text
			{}                  % Indent amount
			{\bfseries}         % Theorem head font
			{:}                 % Punctuation after theorem head
			{\newline}          % Space after theorem head
			{}                  % Theorem head spec (can be left empty, meaning 'normal')
\theoremstyle{algorithms}
\swapnumbers
\newtheorem{algorithm}{Algorithmus}
% Change numbering of algorithms to include chapter
\renewcommand{\thealgorithm}{A\arabic{algorithm}}
%%%% Tables and figures
\numberwithin{table}{section}
\numberwithin{figure}{section}



\input{_preamble/text.tex}
%\input{_preamble/bibtex_only.tex}
%\input{_preamble/biblatex_bibtex.tex}
%\input{_preamble/biblatex_biber.tex}

% !! Hyperref before imakeidx !!
\input{_preamble/hyperref.tex}
%\input{_preamble/indicies.tex}



\author{Johannes Hahn}
\title{TITLE}
%\subtitle{}

\hypersetup{
pdfinfo=
	{  
		Title={TITLE},
		Author={Johannes Hahn},
		Keywords={KEYWORDS},
		Subject={SUBJECT}
	}
}


\begin{document}

\maketitle

\section{Some categorial flavour to algebraic notions}

\begin{definition}[Projective and injective objects]
$P\in Ob(\mathsf{A})$ is called \udot{projective} iff for every epimorphism $B\twoheadrightarrow A$ and every morphism $P\to A$ there is a morphism $P\to B$ making the triangle commutative.

Dually $I\in Ob(\mathsf{A})$ is called \udot{injective} iff for every monomorphism $A\hookrightarrow B$ and every morphism $A\to I$ there is a morphism $B\to I$ making the triangle commutative.

\begin{figure}[ht]
\centering
\begin{tabular}{cc}
\begin{tikzpicture}
\node (B) at (0,0) {$B$};
\node (A) at (2,0) {$A$};
\node (null) at (3,0) {$0$};
\node (P) at (2,2) {$P$};

\path[->]
(A) edge (null)
(P) edge (A)
(P) edge[dotted] (B);

\path[->>]
(B) edge (A);
\end{tikzpicture}
&
\begin{tikzpicture}
\node (B) at (0,0) {$B$};
\node (A) at (2,0) {$A$};
\node (null) at (3,0) {$0$};
\node (I) at (2,2) {$I$};

\path[<-]
(A) edge (null)
(P) edge (A)
(I) edge[dotted] (B);

\path[left hook->]
(A) edge (B);
\end{tikzpicture}
\end{tabular}
\end{figure}
\end{definition}

\begin{remark}
In both cases, the morphisms need not be unique and in many cases they aren't.
\end{remark}

\section{Some homological algebra}

\begin{definition}[Chain complexes]
Let $\mathsf{A}$ be an additive category. A \udot{chain complex} $(A_\ast,\partial)$ is a pair consisting of a graded object $A_\ast\in\mathsf{A}^\IN$ and a morphism $\partial: A\to A$ of degree $-1$, i.e. $\partial_n : A_n \to A_{n-1}$, such that $\partial\circ\partial=0$.

The category of chain complexes is denoted $Ch(\mathsf{A})$.
\end{definition}

\begin{definition}[Homology]
Let $\mathsf{A}$ be an abelian category and $A_\ast\in Ch(\mathsf{A})$ a chain complex. Then its \udot{homology} is defined to be the graded object $H_n := \underbrace{\ker(\partial_n)}_{=:Z_n} / \underbrace{\im(\partial_{n+1})}_{=:B_n}$.
\end{definition}

\subsection{Mapping cone}

\begin{definition}
Let $f:(A_\ast,\partial^A)\to(B_\ast,\partial^B)$ be a chain-map. The \udot{mapping cone} $C(f)=(C(f)_\ast,\partial^{C(f)})$ is the chain complex given by
\[C(f)_n := A_{n-1} \oplus B_n \quad\text{and}\quad \partial_n^{C(f)} := \begin{pmatrix}-\partial_{n-1}^A & 0 \\ -f_{n-1} & \partial_n^B \end{pmatrix}\]
\end{definition}

\begin{lemma}[Mapping cones vs. quasi-isomorphisms]
Let $f:(A_\ast,\partial^A)\to(B_\ast,\partial^B)$ be a chain-map.
\begin{enumerate}
\item $0 \to B \xhookrightarrow{i} C(f) \xtwoheadrightarrow{q} A[-1] \to 0$ is a short exact sequence of chain complexes.
\item The induced long exact sequence in homology
\[\cdots \to H_{n+1}(B) \xrightarrow{i_\ast} H_{n+1}(C(f)) \xrightarrow{q_\ast} \underbrace{H_{n+1}(A[-1])}_{=H_n(A)} \xrightarrow{\delta} H_n(B) \to H_n(C(f)) \to \cdots\]
has $f_\ast$ as connecting morphism $\delta$.
\item $f$ quasi-isomorphism $\iff C(f)$ is acylic.
\item TFAE:
\begin{enumerate}
\item $H_\ast(f)=0$
\item $i_\ast: H_\ast(B) \to H_\ast(C(f))$ is mono.
\item $0\to H_\ast(B) \xrightarrow{i_\ast} H(C(f)) \xrightarrow{q_\ast} H_{\ast-1}(A)\to 0$ is a short exact sequence.
\item $q_\ast: H_\ast(C(f)) \to H_{\ast-1}(A)$ is epi.
\end{enumerate}
\end{enumerate}
\end{lemma}

\begin{lemma}[Mapping cones vs. chain homotopy]
Let $f:(A_\ast,\partial^A)\to(B_\ast,\partial^B)$ be a chain-map.
\begin{enumerate}
\item $f$ is a homotopy-equivalence $\iff C(f)$ is contractible.
\item TFAE:
\begin{enumerate}
\item $f$ is null-homotopic.
\item $f$ factors through $A\hookrightarrow C(\id_A)$.
\item $f$ factors through a contractible complex.
\item The short exact sequence $0\to B\hookrightarrow C(f)\to A[-1]\to 0$ splits.
\end{enumerate}
\end{enumerate}
\end{lemma}
\begin{proof}
a. A map
\[H:=\begin{pmatrix}
\alpha & \beta \\ \gamma & \delta
\end{pmatrix} : \begin{array}{c}A_{n-1}\\\oplus\\B_n\end{array} \to \begin{array}{c}A_n \\\oplus\\ B_{n+1}\end{array}\]
is a homotopy $\id_{C(f)} \homotopic 0$ iff $H\partial^{C(f)} + \partial^{C(f)} H = \id$, that is iff
\[-\begin{pmatrix}
\alpha\partial + \beta f + \partial\alpha & -\beta\partial + \partial\beta \\
\gamma\partial + \delta f + f\alpha-\partial\gamma & -\delta\partial + f\beta - \partial\delta
\end{pmatrix} = \begin{pmatrix}
\id & 0 \\ 0 & \id
\end{pmatrix}\]
i.e. iff $\beta: B\to A$ is a chain-map, $-\alpha$ is a homotopy $\id\homotopic(-\beta) f$, $\delta$ is homotopy $\id\homotopic f(-\beta)$ and $\gamma$ is some map satisfying the last equation.

This already proves one direction: If $C(f) \homotopic 0$, then $f(-\beta) \homotopic \id$ and $(-\beta) f \homotopic\id$ so that $A\homotopic B$.

\medbreak
Conversely, if $f(-\beta) \homotopic \id$ and $(-\beta)f \homotopic \id$ via homotopies $\delta$ and $-\alpha$ respectively, then setting $\gamma:=0$ for the moment, we instead get a homotopy $\tilde{H}$ of $0$ with $\psi:=\begin{pmatrix}
\id & 0 \\ \delta f + f\alpha & \id
\end{pmatrix}$ which is obviously an isomorphism on the level of modules. $\psi$ is in fact a chain map:
\[\partial^{C(f)} \psi - \psi \partial^{C(f)} = \begin{pmatrix}
0 & 0 \\ \partial\delta f + \partial f\alpha + \delta f\partial + f\alpha\partial & 0
\end{pmatrix}\]
This is zero because
\[\partial\delta f + \delta f \partial = \partial\delta f + \delta \partial f = (-\id+f\beta)f\]
and
\[\partial f \alpha + f\alpha\partial = f\partial\alpha+f\alpha\partial = f(\id-\beta f)\]
Therefore $H:=\psi^{-1}\tilde{H}$ is a homotopy $0\homotopic \id_{C(f)}$ so that $C(f)$ is contractible as claimed.

\medbreak
b. i.$\impliedby$ii.$\impliedby$iii. is trivial. We show i.$\iff$iii.: $f$ factorises over the inclusion $A\hookrightarrow C(\id_A), a\mapsto (0,a)$ iff there is a chain-map
\[(h_{n-1}, f_n) : \begin{array}{c}A_{n-1}\\\oplus\\A_n\end{array} \to B_n\]
And for a family $(h_n: A_n\to B_{n+1})$ to induce such a chain map $C(\id_A)\to B$ is equivalent to $\partial^B h =-h\partial^A - f$, i.e. to $h$ being a homotopy $f \homotopic 0$.

i.$\iff$iv.: $B\hookrightarrow C(f)$ splits iff there is a chain map
\[(r,\id) : \begin{array}{c}A_{n-1}\\\oplus\\B_n\end{array} \to B_n\]
And for a family $(r_n: A_n\to B_{n+1})$ to induce such a chain map $C(f)\to B$ is equivalent to $\partial r=-r\partial-f$, i.e. to $r$ being a homotopy $f\homotopic 0$.
\end{proof}

\begin{lemma}[Universal properties of cones]
Let $f:A_\ast\to B_\ast$ be a chain map. Then
\[\Hom_{Ch}(X_\ast, Cone(f)) = \Set{\begin{pmatrix}\gamma\\h\end{pmatrix} | X \xrightarrow{\gamma} A[-1], X\xrightarrow{h}B, f[1]\circ\gamma \overset{h}{\homotopic} 0}\]
\[\Hom_{Ch}(Cone(f),Y_\ast) = \Set{(h,\beta) | A[-1]\xrightarrow{h}Y, B \xrightarrow{\beta} Y, \beta \circ f \overset{h}{\homotopic} 0}\]
\end{lemma}

\subsection{Replacing objects by projective / injective resolutions}

\begin{lemma}
Chain maps between projectives / acyclic complexes are unique up to homotopy:
\begin{multicols}{2}
\begin{enumerate}
\item Homology: If $C_\ast\in Ch^-(\mathsf{A})$ is acyclic and $P_\ast\in Ch^-(Proj(\mathsf{A}))$ all morphisms $P_\ast\to C_\ast$ are null-homotopic.
\item Cohomology: If $C^\ast\in Ch^+(\mathsf{A})$ is acyclic and $I^\ast\in Ch^+(Inj(\mathsf{A}))$ all morphisms $C^\ast \to I^\ast$ are null-homotopic.
\end{enumerate}
\end{multicols}
\end{lemma}
\begin{proof}
Let $\alpha: P_\ast\to C_\ast$ be a chain map. Inductively we construct a homotopy $h:P_\ast\to C_\ast[1]$ to the zero map.

\begin{tikzcd}[row sep=large, column sep=large]
\cdots \arrow[r] &
P_2 \ar[r] \ar[d, "\alpha_2" description] \ar[dl, dotted, "h_2" description] &
P_1 \ar[r] \ar[d, "\alpha_1" description] \ar[dl, dotted, "h_1" description] &
P_0 \ar[r] \ar[d, "\alpha_0" description] \ar[dl, dotted, "h_0" description] & 0 \\
\cdots \ar[r] & C_2 \ar[r] & C_1 \ar[r] & C_0 \ar[r] & 0
\end{tikzcd}

We begin with setting $h_n:=0$ for all $n<0$. First step is to construct $h_0$. Since $C$ is exact, $C_1\to C_0$ is epi so that $\alpha_0$ lifts to some $h_0: P_0 \to C_1$ by projectivity, so that $\partial_1 h_0 + \partial 0 = \alpha_0$ is satisfied.

If $h_0,\ldots,h_{n-1}$ are already known and a partial homotopy, then
\begin{align*}
\partial_n\alpha_n &= \alpha_{n-1}\partial_n \\
&=(\partial_n h_{n-1} + h_{n-2} \partial_{n-1})\partial_n \\
&=\partial_n h_{n-1} \partial_n
\end{align*}
So that $\partial(\alpha_n - h_{n-1}\partial_n)=0$. Therefore $\alpha_n-h_{n-1}\partial_n$ maps into $Z_n(C)$ which equals $B_n(C)=\im(\partial_{n+1})$ by exactness. By projectivity, we can find $h_n$ such that
\[\alpha_n - h_{n-1}\partial_n = \partial_{n+1} h_n\]
is satisfied which proves the lemma.
\end{proof}

\begin{corollary}[Fundamental lemma of homological algebra]
\enquote{Objects can be replaced by their projective or injective resolutions}
\begin{multicols}{2}
\begin{enumerate}
\item Homology: Assume that $\mathsf{A}$ has enough projectives and that a projective resolution has been fixed for every object.

Any $f:A\to B$ extends to a chain map between the augmented complexes
\[\begin{tikzcd}
P_\ast(A) \ar[r] \ar[d, "\phi" description] & A \ar[r] \ar[d, "f" description] & 0\\
P_\ast(B) \ar[r] & B \ar[r] & 0
\end{tikzcd}\]
$\phi$ is unique up to homotopy.

In particular: $\mathsf{A} \xrightarrow{P_\ast} K^-(Proj(\mathsf{A}))$ is a well-defined functor with $H_0\circ P_\ast \isomorphic \id_\mathsf{A}$.

\item Cohomology: Assume that $\mathsf{A}$ has enough injectives and that a injective resolution has been fixed for every object.

Any $f:A\to B$ extends to a chain map between the augmented complexes
\[\begin{tikzcd}
0 \ar[r] & A \ar[r] \ar[d, "f" description] & I^\ast(A) \ar[d, "\phi" description] \\
0 \ar[r] & B \ar[r] & I^\ast(B) 
\end{tikzcd}\]
$\phi$ is unique up to homotopy.

In particular: $\mathsf{A} \xrightarrow{I^\ast} K^+(Inj(\mathsf{A}))$ is a well-defined functor with $H_0\circ I^\ast \isomorphic \id_\mathsf{A}$.
\end{enumerate}
\end{multicols}
As a consequence, projective and injective resolutions are unique up to homotopy equivalence.
\end{corollary}
\begin{proof}
Uniqueness up to homotopy follows from the lemma. We only have to show existence. Again, we work inductively:
\[\begin{tikzcd}[row sep=large]
\cdots \ar[r] &
P_2(A) \ar[r] \ar[d, dotted, "\phi_2" description] &
P_1(A) \ar[r] \ar[d, dotted, "\phi_1" description] &
P_0(A) \ar[r] \ar[d, dotted, "\phi_0" description] &
A \ar[r] \ar[d, "f" description] & 0\\
\cdots \ar[r] & P_2(B) \ar[r] & P_1(B) \ar[r] & P_0(B) \ar[r] &
B \ar[r] & 0
\end{tikzcd}\]

We set $P_{-1}(A):=A$, $\phi_{-1}:=f$,  and $P_{-1}(B):=B$ for notational convenience. If $\phi_{n-1}$ is already constructed, then
\[\begin{tikzcd}
P_n(A) \ar[r, "\partial"] & P_{n-1}(A) \ar[d, "\phi_{n-1}"] & \\
 & P_{n-1}(B) \ar[r, "\partial"] & P_{n-2}(B)
\end{tikzcd} = \begin{tikzcd}
P_n(A) \ar[r, "\partial"] & P_{n-1}(A) \ar[r, "\partial"] & P_{n-2}(A) \ar[d, "\phi_{n-2}"] \\
 & & P_{n-2}(B)
\end{tikzcd} = 0\]
Therefore $\phi_{n-1}\circ\partial_n: P_n(A) \to P_{n-1}(B)$ maps into $Z_{n-1}(P_\ast(B))$ which equals $B_{n-1}(P_\ast(B)) = \im(\partial_n)$ by exactness. By projectivity, we get a lift $\phi_n: P_n(A) \to P_n(B)$.
\end{proof}


\begin{lemma}[Horseshoe lemma]
\enquote{$P_\ast$ and $I^\ast$ are exact}

\begin{multicols}{2}
\begin{enumerate}
\item Homology: Every diagram
\[\begin{tikzcd}[column sep=tiny, row sep=small]
& P_\ast(A) \ar[d] & & P_\ast(C) \ar[d] & \\
0 \ar[r] & A \ar[r]\ar[d] & B \ar[r] & C \ar[r]\ar[d] & 0 \\
& 0 & & 0 &
\end{tikzcd}\]
with exact first row and projective resolutions in the columns can be extended with some projective resolution $P_\ast(B)\to B\to 0$ to a diagram
\[\begin{tikzcd}[column sep=tiny, row sep=small]
0 \ar[r] & P_\ast(A) \ar[r] \ar[d] & P_\ast(B) \ar[r]\ar[d] & P_\ast(C) \ar[r]\ar[d] & 0 \\
0 \ar[r] & A \ar[r]\ar[d] & B \ar[r]\ar[d] & C \ar[r]\ar[d] & 0 \\
& 0 & 0 & 0 &
\end{tikzcd}\]
in which all rows are exact.

\item Cohomology: Every diagram
\[\begin{tikzcd}[column sep=tiny, row sep=small]
& 0 \ar[d] & & 0 \ar[d] & \\
0 \ar[r] & A \ar[r] \ar[d] & B \ar[r] & C \ar[r] \ar[d] & 0\\
& I^\ast(A) & & I^\ast(C)&
\end{tikzcd}\]
with exact first row and injective resolutions in the columns can be extended with some injective resolution $0\to B\to I^\ast(B)$ to a diagram
\[\begin{tikzcd}[column sep=tiny, row sep=small]
& 0 \ar[d] & 0\ar[d] & 0\ar[d] & \\
0 \ar[r] & A \ar[r] \ar[d] & B \ar[r]\ar[d] & C \ar[r]\ar[d] & 0\\
0 \ar[r] & I^\ast(A) \ar[r] & I^\ast(B) \ar[r] & I^\ast(C) \ar[r] & 0
\end{tikzcd}\]
in which all rows are exact.
\end{enumerate}
\end{multicols}
\end{lemma}
\begin{proof}
Set $A_{-1}:=A$ and $A_n:=P_n(A)$, $C_{-1}:=C$ and $C_n:=P_n(C)$ as well as $B_{-1}:=B$. Then define $P_n(B) := B_n:=A_n\oplus C_n$.

For the vertical maps consider
\[\begin{tikzcd}
0 \ar[r] & A_n \ar[r] \ar[d] & A_n\oplus C_n \ar[r] \ar[d,dotted,"g" description] & C_n \ar[r] \ar[d] \ar[dl,dotted,"h" description] \ar[dll,dotted,"f"' near end] & 0 \\
0 \ar[r] & A_{n-1} \ar[r] \ar[d] & B_{n-1} \ar[r]\ar[d] & C_{n-1} \ar[r]\ar[d] & 0 \\
0 \ar[r] & A_{n-2} \ar[r] & B_{n-2} \ar[r] & C_{n-2} \ar[r] & 0 \\
\end{tikzcd}\]
We define $g:A_n\oplus C_n\to B_{n-1}$ separately on the two components. Define $g:A_n\oplus 0\to B_{n-1}$ to be the composition $A_n\to A_{n-1}\to B_{n-1}$.

The map $g: 0\oplus C_n\to B_{n-1}$ we choose in two steps. First choose $h: C_n \to B_{n-1}$ to make the triangle on the right side commute. Then
\[\begin{tikzcd}[column sep=tiny, row sep=small]
0\oplus C_n \ar[d,"h"] & \\
B_{n-1} \ar[d] & \\
B_{n-2} \ar[r] & C_{n-2} & \\
\end{tikzcd} = \begin{tikzcd}[column sep=tiny, row sep=small]
0\oplus C_n \ar[r,"="] & C_n \ar[d] \\
& C_{n-1} \ar[d] \\
& C_{n-2} \\
\end{tikzcd} = 0\]
d.h. $\partial h(C_n) \subseteq A_{n-2}$ because the $(n-2)$th row is exact and of course $\partial\partial h=0$ so that $\partial h(C_n) \subseteq Z_{n-2}(A_\ast) = B_{n-2}(A_\ast)$ by exactness of $A_\ast$. Using projectivity once again, we can lift $\partial h$ to $f: C_n \to A_{n-1}$ and finally define $g:0\oplus C_n\to B_{n-1}$ as $h-f$. Note that $\overline{g(c_n)} = \partial c_n$ still holds because $\im(f) \subseteq\ker(B_{n-1}\to C_{n-1})$.

\medbreak
This ensures $\partial g = 0$ which proves that the middle column is a(n incomplete) complex. We still have to show exactness. So let $b_{n-1}\in B_{n-1}$ with $\partial b_{n-1}=0$.

Then its image $c_{n-1}=\overline{b_{n-1}}$ also satisfies $\partial c_{n-1} =0$ so that a $c_n$ exists with $c_{n-1} = \partial c_n$ by exactness of $C_\ast$. Then $\overline{b_{n-1}-g(0\oplus c_n)} = c_{n-1} - \partial c_n = 0$ so that $b_{n-1}-g(0\oplus c_n)\in\ker(B_{n-1} \to C_{n-1})$ which is $\im(A_{n-1} \to B_{n-1})$ by exactness of the $(n-1)$th row so that $b_{n-1}-g(0\oplus c_n) = a_{n-1}$. Then $0=0-0=\partial b_{n-1} - \partial g(0\oplus c_n) = \partial a_{n-1}$ so that $a_{n-1} = \partial a_n = g(a_n \oplus 0)$. That shows $b_{n-1} = g(a_n\oplus c_n)$.
\end{proof}

\subsection{Replacing complexes by projective / injective resolutions}

\begin{corollary}
\enquote{Complexes can be replaces by double complexes of projectives/injectives}
\begin{multicols}{2}
\begin{enumerate}
\item Homology: For every $K_\ast\in Ch(\mathsf{A})$ exists a commutative double complex $P_{\ast,\ast}$ and maps $P_{n,\ast}\to K_n$ such that $P_{\ast,n} \to K_n \to 0$ is a projective resolution.
\item Cohomology: For every $K^\ast\in Ch(\mathsf{A})$ exists a commutative double complex $I^{\ast,\ast}$ and maps $K^n\to I^{n,\ast}\to K^n$ such that $0\to K^n\to I^{n,\ast} \to K^n \to 0$ is an injective resolution.
\end{enumerate}
\end{multicols}
\end{corollary}
\begin{proof}
Consider the short exact sequences
\[0\to Z_n \to K_n \to B_{n-1}\to 0\]
and choose projective resolutions $P_{n,\ast}''\to Z_n\to 0$ and $P_{n,\ast}'\to B_n\to 0$. Apply the horseshoe lemma to obtain a projective resolution $P_{n,\ast}\to K_n\to 0$ fitting in the exact sequence.

Now apply the fundamental lemma of homological algebra to get a chain-map  $P_{n,\ast}' \to P_{n-1,\ast}''$ that extends the canonical map $B_{n-1} \to Z_{n-1}$ and let $P_{n,\ast} \to P_{n-1,\ast}$ be the composition $P_{n,\ast} \twoheadrightarrow P_{n,\ast}' \to P_{n-1,\ast}'' \hookrightarrow P_{n-1,\ast}$. Since $P' \to P \to P''$ are short exact sequences, we obtain a commutative double complex in this way.
\end{proof}

\begin{lemma}
\enquote{Projective / injective resolutions of complexes exist}

\begin{multicols}{2}
\begin{enumerate}
\item Homology: For any bounded above complex $K_\ast\in Ch^-(\mathsf{A})$ there is a $P_\ast \in Ch^-(Proj(\mathsf{A}))$ and a quasi-isomorphism $P_\ast \to K_\ast$.

$P_\ast$ can be chosen such that the quasi-isomorphism is termwise epi: $P_n\twoheadrightarrow K_n$.

\item Cohomology: For any bounded below complex $K^\ast\in Ch^+(\mathsf{A})$ there is a $I^\ast \in Ch^+(Inj(\mathsf{A}))$ and a quasi-isomorphism $K^\ast \to I^\ast$.

$I_\ast$ can be chosen such that the quasi-isomorphism is termwise mono: $K_n \hookrightarrow I_n$.
\end{enumerate}
\end{multicols}
\end{lemma}
\begin{proof}
Take the total complex of $P_{\ast,\ast}$ in the previous statement.
\end{proof}

\begin{lemma}
\enquote{Fundamental lemma of homological algebra upgraded to complexes}

Let $A_\ast,Q_\ast \in Ch^-(\mathsf{A}), P_\ast\in Ch^-(Proj(\mathsf{A}))$ be quasi-isomorphic, say $Q_\ast \xrightarrow[\sim]{\alpha} A_\ast$ and $P_\ast \xrightarrow[\sim]{\beta} A_\ast$.

If $Q_n \xtwoheadrightarrow{\alpha} A_n$ is termwise epi, then there exists a chain-map $P_\ast \xrightarrow{\gamma} Q_\ast$ such that $\alpha\circ\gamma = \beta$.

If $\alpha$ is arbitrary, there exists a $\gamma$ s.t. $\alpha\circ\gamma \homotopic \beta$.

Any two chain-maps with $\alpha\circ\gamma_1\homotopic\beta\homotopic\alpha\circ\gamma_2$ are homotopic.
\end{lemma}
\begin{remark}
If $A$ is concentrated in a single degree, then $Q_\ast \to A_0\to 0$ is just an acyclic complex and the statement reduces to the fundamental lemma of homological algebra. In this sense this statement is a generalisation of the fundamental lemma from $\mathsf{A}$ to $D^-(\mathsf{A})$.
\end{remark}

\begin{proof}[Homological version, seems harder??]
Assume that a partial chain map $\gamma_0,\ldots,\gamma_{n-1}$ is already constructed. We want to construct the missing arrow in the commutative diagram
\[\begin{tikzcd}
P_n \ar[rr] \ar[d,dotted] \ar[ddd,bend right] && P_{n-1} \ar[d] \ar[ddd,bend left] \\
Q_n \ar[dd,->>] \ar[rr] \ar[rd] && Q_{n-1} \ar[dd,->>] \\
& F \ar[ru] \ar[ld] & \\
A_n \ar[rr] && A_{n-1}
\end{tikzcd}\]
We set
\[F:=A_n \times_{A_{n-1}} Z(Q_{n-1}) = \Set{(a_n,q_{n-1})\in A_n\times Q_{n-1} | \partial q_{n-1} = 0 \wedge \partial a_n = \alpha(q_{n-1})}\]
First we prove that the map $Q \xrightarrow{(\alpha,\partial)} F$ is epi. Let $(a_n,q_{n-1})\in F$ be arbitrary.

Then $q_{n-1} \in Z_{n-1}(Q)$ so that the homology class is well-defined. Then $\alpha_\ast [q_{n-1}]_{H_{n-1}(Q)} = [\alpha(q_{n-1})]_{H_{n-1}(A)} = [\partial a_n] = 0$. Since $\alpha$ is injective on homology, this means $[q_{n-1}] = 0$, i.e. $q_{n-1} = \partial q_n'$ for some $q_n'\in Q_n$.

Then $\partial a_n = \alpha(q_{n-1}) = \alpha\partial(q_n') = \partial \alpha(q_n')$ so that $a_n-\alpha(q_n')\in Z_n(A)$. Since $\alpha$ is surjective on homology, there is a $z_n\in Z_n(Q)$ such that $[\alpha(z_n)] = [a_n-\alpha(q_n')]$, i.e. there exists a $a_{n+1}$ such that $\alpha(z_n) = a_n - \alpha(q_n') + \partial a_{n+1}$.

Now choose an preimage $q_{n+1}\in Q_{n+1}$ of $a_{n+1}$ and set $q_n := z_n+q_n' - \partial q_{n+1}$. This is the preimage of $(a_n,q_{n-1})$:

\[\alpha(q_n) = \underbrace{\alpha(z_n)+\alpha(q_n')}_{=a_n+b_n} - \alpha(\partial q_{n+1}) = a_n + b_n - \partial\alpha(q_{n+1}) = a_n\]
\[\partial(q_n) = \underbrace{\partial z_n}_{=0} + \underbrace{\partial q_n'}_{=q_{n-1}} + 0\]

\medbreak
Since we now know that $Q_n \to F$ is epi, we can lift the morphism $(\beta_n,\gamma_{n-1}\partial): P_n \to F$ to a morphism $\gamma_n: P_n\to Q_n$. By construction it makes the diagram commute so that it is a partial chain map.

\medbreak
b. If $\alpha$ is not term-wise epi \textcolor{red}{Meep}

\medbreak
c. For uniqueness observe that $\alpha\circ(\gamma_1-\gamma_2) \homotopic 0$ so that there is a chain-map $P_\ast[-1] \xrightarrow{(h,\gamma_1-\gamma_2)} Cone(\alpha), p \mapsto (h(p), (\gamma_1-\gamma_2)(p))$. Since $\alpha$ is a quasi-isomorphism, $Cone(\alpha)$ is acyclic so that any such map is null homotopic. In particular $\gamma_1-\gamma_2 = quotient\circ \widehat{\gamma} \homotopic 0$.
\end{proof}

\begin{corollary}[Resolution functor]
Fixing a projective resolution of every complex, $P_\ast: D^-(\mathsf{A}) \to K^-(Proj(\mathsf{A}))$ is a pseudo-inverse to the localisation functor.
\end{corollary}
\begin{proof}
We have to show $\Hom_{K}(P_\ast(A_\ast) , P_\ast(B_\ast)) = \Hom_D(A_\ast,B_\ast)$.

Morphisms $A\xrightarrow[\gamma]{} B$ in $D^-$ are roofs $A \to M \xleftarrow[\sim]{} B$.
\[\begin{tikzcd}
P_\ast(A) \ar[rr,dotted,bend left] \ar[d] & M & P_\ast(B) \ar[d] \\
A \ar[rr,dashed] \ar[ru] && B \ar[lu]
\end{tikzcd}\]
Since $P_\ast(A)$ is termwise projective and $P_\ast(B) \xrightarrow[\sim]{}B \xrightarrow[\sim]{} M$ is a quasi-isomorphism, we can complete the triangle of $P_\ast(A) \to M$ and $M\xleftarrow[\sim]{} P_\ast(B)$ with a chain-map $\gamma: P_\ast(A) \to P_\ast(B)$ making the diagram commute up to homotopy.
\end{proof}

\section{Derived functors I}

\begin{remark}[The Problem]
Given a right-exact functor $F: \mathsf{A} \to \mathsf{B}$, and exact sequence
\[0\to A \to B \to C \to 0\]
gives a exact sequence
\[F(A) \to F(B) \to F(C) \to 0\]
We want to find functors $L_nF$ and natural transformations $\delta_n$ (natural w.r.t. the short exact sequence) such that this sequence extends to a long exact sequence
\[\cdots \to L_2F(C) \xrightarrow{\delta_2} L_1F(A) \to L_1F(B) \to L_1F(C) \xrightarrow{\delta_1} \underbrace{F(A)}_{=L_0F(A)} \to \underbrace{F(B)}_{L_0F(B)} \to \underbrace{F(C)}_{=0} \to 0\]
And similarly for left-exact functors.

Of course, we want the universal solution to this problem.
\end{remark}

\begin{definition}[$\delta$-functors]
A family $(F_n,\delta_n)_{n\in\IN}$ of functors $\mathsf{A} \xrightarrow{F_n} \mathsf{B}$ and natural transformations $F_n(C) \xrightarrow{\delta_n} F_{n-1}(A)$ for every short exact sequence $0\to A\to B\to C\to 0$ that transforms such short exact sequences into long exact sequences as above is called a \udot{(homological) $\delta$-functor}.

A morphism $F\xrightarrow{t}G$ of $\delta$-functors is a family $(t_n)$ of natural transformations $F_n \xrightarrow{t_n} G_n$ which induces morphisms between the long exact sequences, i.e. $t_{n-1}\delta_n^F = \delta_n^G t_n$.

Cohomological $\delta$-Functors $(F^n,d^n)$ are analogously defined.
\end{definition}

\begin{definition}[Universal $\delta$-functors]
A $\delta$-functor $(F_n,\delta_n)$ is called the \udot{universal $\delta$-functor} if for every $(G_n,\delta_n)$ and every $G_0 \xrightarrow{t_0} F_0$ there exists a unique morphism $G \xrightarrow{t} F$ of $\delta$-functors extending $t_0$.
\end{definition}

\begin{definition}[Derived functors]
Let $F: \mathsf{A} \to \mathsf{B}$ be right-exact. A $\delta$-functor $(L_nF,\delta_n)$ together with an isomorphism $L_0 F \xrightarrow{\tau} F$ is called the \udot{left derived functor of $F$} if $(LF,\tau)$ is a final object in the category of all $\delta$-functor-with-isomorphisms.

It is in other words a representation of the functor $\set{\delta\text{-functors}} \to \mathsf{Set}, (G_n,\delta_n)\mapsto \operatorname{Nat}(G_0,F)$ such that the universal element $\tau\in Nat(F_0,F)$ is an iso.

\medbreak
Similarly \udot{right derived functor} of a left exact $F$ is defined as an initial object in the appropriate category of $\delta$-functors with isomorphism $F\xrightarrow[\isomorphic]{\tau} F_0$, i.e. a representation of the functor $(G^n,d^n) \mapsto \operatorname{Nat}(F,G_0)$.
\end{definition}

\begin{remark}
In particular: Derived functors are exactly those universal $\delta$-functors that agree with $F$ in degree $0$.
\end{remark}

\begin{lemma}[Recognising universal $\delta$-functors]
Let $(F_n,\delta_n)$ be a $\delta$-functor.
\begin{multicols}{2}
\begin{enumerate}
\item Homology: If $F_n(P) = 0$ for all $n\geq 1$ and all $P\in Proj(\mathsf{A})$, then $F$ is a universal homological $\delta$-functor.
\item Cohomology: Assume $F^n(I) = 0$ for all $n\geq 1$ and all $I\in Inj(\mathsf{A})$, then $F$ is a universal cohomological $\delta$-functor.
\end{enumerate}
\end{multicols}
\end{lemma}
\begin{proof}
Let $(\tilde{F}_n,\tilde{\delta}_n)$ be another $\delta$-functor and assume that unique transformations $t_0,\ldots,t_{n-1}$ have already been constructed. Fix $C\in\mathsf{A}$ and choose a short exact $0\to K\xrightarrow{j} P \xrightarrow{q} C\to 0$ with $P$ projective. Then
\[\begin{tikzcd}
\cdots \ar[r] & \tilde{F}_n(P) \ar[r]\ar[d,dotted]  & \tilde{F}_n(C) \ar[r,"\tilde{\delta_n}"]\ar[d,dotted]  & \tilde{F}_{n-1}(K) \ar[r]\ar[d,"t_{n-1}"]  & \tilde{F}_{n-1}(P)\ar[d,"t_{n-1}"] \ar[r] & \cdots \\
\cdots \ar[r] & \underbrace{F_n(P)}_{=0} \ar[r] & F_n(C) \ar[r,"\delta_n"] & F_{n-1}(K) \ar[r] & F_{n-1}(P) \ar[r] & \cdots 
\end{tikzcd}\]
It follows that $F_n(C) \xrightarrow[\isomorphic]{\delta_n} \ker(F_{n-1}(j))$ and since $t_{n-1}$ is natural, there is a unique $t_n: \tilde{F}_n(C) \to F_n(C)$ that makes the square commute. This $t_n$ does not depend on the choice of $K$ and $P$ by Schanuel's lemma.

Naturality of $t_n$ follows from a simple diagram chase using naturality of $\delta_n$ and $\tilde{\delta_n}$, naturality of $t_{n-1}$ and that $F_n(A) \to F_{n-1}(K)$ is mono.

It remains to show that $t_n$ commutes with the deltas for an arbitrary short exact $0\to A\to B\to C\to 0$. This also follows from a simple diagram chase.
\end{proof}

\begin{theorem}[Derived functors exist.]
Let $F:\mathsf{A}\to\mathsf{B}$ be right / left exact functor.
\begin{multicols}{2}
\begin{enumerate}
%\begin{minipage}{0.45\linewidth}
\item Homology:
\begin{enumerate}
\item If $\mathsf{A}$ has enough projectives, then $F$ has a left derived functor.
\item $L_i F(P) = 0$ for all projectives $P$ and all $i>0$.
\item Deriving is a functor $L_i: \mathsf{Fun}_\text{r.e.}(\mathsf{A},\mathsf{B}) \to \mathsf{Fun}_\text{add}(\mathsf{A},\mathsf{B})$.
\end{enumerate}
%\end{minipage}
%\begin{minipage}{0.45\linewidth}
\item Cohomology:
\begin{enumerate}
\item If $\mathsf{A}$ has enough injectives, then $F$ has a right derived functor.
\item $R^i F(I) = 0$ for all injectives $I$ and all $i>0$.
\item Deriving is a functor $R^i: \mathsf{Fun}_\text{l.e.}(\mathsf{A},\mathsf{B}) \to \mathsf{Fun}_\text{add}(\mathsf{A},\mathsf{B})$.
\end{enumerate}
%\end{minipage}
\end{enumerate}
\end{multicols}
\end{theorem}
\begin{proof}
Existence: Define $L_nF := H_n \circ P_\ast$. Note that this does not depend on the choice of the projective resolutions $P_\ast$ because all choices are homotopy equivalent and homology forgets homotopy. Note that $L_i F(P) = 0$ for $P$ projective and $i>0$ because $0\to P\xrightarrow{\id} P \to 0$ is a projective resolution of $P$.

\medbreak
Horseshoe lemma implies that every short exact sequence
\[0\to A\to B\to C\to 0\]
lifts to exact sequence up to homotopy $0\to P_\ast(A) \to P_\ast(B) \to P_\ast(C)\to 0$ which implies a long exact sequence in homology. Therefore $LF=(L_i F,\delta_i)$ is a $\delta$-functor. It extends $P$ because $P_\ast(A)\to A\to 0$ is a projective resolution such that $H_0(P_\ast(A)) \isomorphic A$ naturally.

\medbreak
We still have to show universality. Let $((\tilde{F}_n,\tilde{\delta}_n), \tilde{\tau})$ be another $\delta$-functor extending $F$. The above lemma shows that there is a unique morphism of $\delta$-functors $t: \tilde{F}\to F$ which extends $t_0 := \tau^{-1}\circ\tilde{\tau}$.

\medbreak
The lemma also proves that every natural transformation $F\to G$ between right exact functors extends to $LF \to LG$ since $L_iG(P) = 0$.
\end{proof}

\subsection{Computing derived functors via acyclic resolutions}

\begin{definition}[$F$-acyclic objects]
An object $Q\in\mathsf{A}$ is called \udot{$F$-acyclic} if
\begin{enumerate}
\item Homology: $L_n F(Q) = 0$
\item Cohomology: $R^n F(Q) = 0$
\end{enumerate}
holds for all $n\geq 1$.
\end{definition}

\begin{remark}
$Proj(A) \subseteq Acycl(F)$ for all $F$. For some $F$ (like $\Hom(A,-)$ equality may hold), but depending on $F$, the class of acyclics may be bigger then the projectives (or injectives). For example all $Proj(A\mathsf{-Mod}) \subsetneq Flat(A\mathsf{-Mod}) \subseteq Acycl(M\otimes-)$.
\end{remark}

\begin{theorem}
Let $F:\mathsf{A}\to\mathsf{B}$ be right / left exact.

\begin{enumerate}
\item Homology: Let $Q_\ast \to A\to 0$ be a resolution of $A$ by $F$-acyclic objects. Then $L_nF(A) \isomorphic H_n(F(Q_\ast))$ via a unique isomorphism.

More precisely: Choose a projective resolution $P_\ast(A) \to A \to 0$. Then the unique-up-to-homotopy chain map $P_\ast(A)\xrightarrow{\gamma} Q_\ast$ induces a unique isomorphism $L_nF(A) = H_n(F(P_\ast(A))) \xrightarrow{H_n(F\gamma)} H_n(F(Q_\ast))$.
\end{enumerate}
\end{theorem}

\begin{lemma}
The class of $F$-acyclics has the following properties:
\begin{enumerate}
\item Homology: Assume $\mathsf{A}$ has enough projectives. Then
\begin{enumerate}
\item Every $A\in\mathsf{A}$ has a covering $Q\twoheadrightarrow A\to 0$ for some acyclic $Q$.
\item It is closed under direct sums.
\item If in an exact sequence $0\to A\to B\to C\to 0$ both $B$ and $C$ are acyclic, then $A$ is two.
\item If in an exact sequence $0\to A\to B\to C\to 0$ the object $C$ is acyclic, then $0\to FA\to FB\to FC\to 0$ is also exact.
\end{enumerate}
\end{enumerate}
\end{lemma}
\begin{proof}
a. Projectives are always acyclic.

b. follows because $L_n F$ is additive.

c. and d. follow from the long exact sequence.
\end{proof}

\begin{lemma}
Let $Q_\ast\in Ch^\pm(Acyc)$ be a complex of $F$-acyclics. If $Q_\ast$ is exact, then $FQ_\ast$ is also exact.
\end{lemma}
\begin{proof}
Let $K_n$ be the kernels / images of the boundary maps so that we get a diagram
\[\begin{tikzcd}
&& K_2 \ar[rd,hook] && && && \\
\cdots \ar[r] & Q_3 \ar[ru,->>] \ar[rr] && Q_2 \ar[dr,->>] \ar[rr] && Q_1 \ar[rr,->>]  && Q_0 \ar[r] & 0 \\
K_3 \ar[ru,hook] && && K_1 \ar[ru,hook] && &&
\end{tikzcd}\]
where the diagonals are short exact sequences. First observation: Per induction all $K_n$ are acyclic, because the $Q_n$ are.

The transformed sequence
\[\begin{tikzcd}
&& FK_2 \ar[rd] && && && \\
\cdots \ar[r] & FQ_3 \ar[ru,->>] \ar[rr] && FQ_2 \ar[dr,->>] \ar[rr] && FQ_1 \ar[rr,->>] && FQ_0 \ar[r] & 0 \\
FK_3 \ar[ru] && && FK_1 \ar[ru] && &&
\end{tikzcd}\]
is exact iff the diagonals are exact again, i.e. if $FK_n \to FQ_n$ is mono. This follows from exactness of $K_n \hookrightarrow Q_n \twoheadrightarrow K_{n-1}$ and $K_{n-1}$ being acyclic.
\end{proof}

\begin{proof}[Proof of the main theorem]
Let $P_\ast \to A$ be a projective resolution, $Q_\ast \to A$ an acylic resolution and $\gamma: P_\ast\to Q_\ast$ be a chain-map extending $A\xrightarrow{\id} A$ along those resolutions. $\gamma$ is a quasi-isomorphism because both resolutions compute have homology $A[0]$.

Therefore $Cone(\gamma)$ is exact. Termwise it is direct sum of projective and $F$-acyclic objects and therefore an exact complex of $F$-acyclic objects. Thus $Cone(F\gamma) = F(Cone(\gamma))$ is also exact. Therefore $F\gamma$ is a quasi-isomorphism.
\end{proof}

\section{Examples}

\begin{example}[Snake lemma]
Taking kernels is a left-exact functor $\mathsf{A}^{\set{\ast\to\ast}} \to \mathsf{Ab}$. Its right derived functor is the cokernel in degree 1 and zero further up.

Dually taking cokernels is right-exact and its left derived functor is the kernel in degree 1 and zero everywhere else.

This is a manifestation of the snake lemma.
\end{example}

\begin{example}[Sheaf (co)homology]
Sheaf cohomology $H^\ast(X,\mathcal{F})$ is the right derived functor of the global section functor $\Gamma: Sh(X) \to \mathsf{Ab}$.
\end{example}

\begin{example}[DeRham cohomology]
$H_\text{dR}^\ast(M)$ is Sheaf cohomology of the sheaf $\underline{\IR}_M$ of locally constant functions $M\to\IR$.

This uses that
\[0 \to \underline{\IR}_M \hookrightarrow \Omega^0(M) \xrightarrow{d} \Omega^1(M) \xrightarrow{d} \cdots \xrightarrow{d} \Omega^n(M) \to 0\]
is a resolution of $\underline{\IR}_M$ by fine sheafs and that fine sheafs are $\Gamma$-acyclic.
\end{example}

\begin{example}[Singular cohomology]
$H_\text{sing}^\ast(X;G)$ is sheaf cohomology of the sheaf $\underline{G}_X\in Sh(X)$ of locally constant $G$-valued functions if $X$ is paracompact.
\end{example}

\begin{example}[Étale cohomology]
Étale cohomology is the Sheaf cohomology for sheafs on the étale site, i.e. the right derived functor of global sections $\mathsf{Sh}_{et}(X) \to \mathsf{Ab}$.
\end{example}

\begin{example}[Ext and Tor]
$\operatorname{Ext}_A^i(M,N)$ is right derived of $\Hom_A(M,-): A\mathsf{-Mod} \to \mathsf{Ab}$ as well as left derived of $\Hom_A(-,N): \mathsf{Mod-}A\to\mathsf{Ab}^{op}$.

$\operatorname{Tor}_i^A(M,N)$ is left derived of both $M\otimes_A -: A\mathsf{-Mod}\to\mathsf{Ab}$ and $ -\otimes_A N:\mathsf{Mod-}A\to\mathsf{Ab}$.
\end{example}

\begin{example}[Group (co)homology]
$H_\ast(G,M)$ is the left derived functor of the functor of coinvariants $k \otimes_{kG} -$, i.e. it is $Tor_\ast^{kG}(k,M)$.

$H_k^\ast(G,-)$ is the right derived functor of the functor of fixed points $(-)^G = \Hom_{kG}(k,-)$, i.e. it is $Ext_{kG}^\ast(k,M)$.
\end{example}

\begin{example}[Hochschild (co)homology]
Let $A^e := A\otimes_k A^{op}$ be the enveloping algebra of the $k$-algebra $A$.

$HH_n(A,M) := Tor_n^{A^e}(A,M)$, i.e. it is the left derived functor of the functors of coinvariant $M/[A,M] = A\otimes_{A^e} M: (A,A)\mathsf{-Bimod} \to \mathsf{Ab}$.

$HH^n(A,M) := Ext_{A^e}^N(A,M)$, i.e. the right derived functor of invariants $Z(M) := \Hom_{A^e}(A,M): (A,A)\mathsf{-Bimod} \to \mathsf{Ab}$.
\end{example}

\begin{example}[Lie-algebra (co)homology]
$H_n(\mathfrak{g},M) := Tor_n^{U(\mathfrak{g})}(k,M)$, i.e. left derived of taking coinvariants.

$H^n(\mathfrak{g},M) := Ext^n_{U(\mathfrak{g})}(k,M)$, i.e. right derived of taking invariants.
\end{example}

\section{Derived functors II: Total derived functors}

\begin{definition}
Let $p_\mathsf{A}^?: K^?(\mathsf{A}) \to D^?(\mathsf{A})$ be the projection functor from the homotopy category onto the derived category.

The total left/right derived functor of $F:\mathsf{A}\to\mathsf{B}$ is the \enquote{best approximation} of $K^\pm(F): K^\pm(\mathsf{A}) \to K^\pm(\mathsf{B})$ on the level of derived categories, i.e. it fits into the diagram
\[\begin{tikzcd}[sep = large]
K^{\pm}(\mathsf{A}) \ar[r,"K^\pm(F)"] \ar[d,"p_\mathsf{A}^\pm"'] & K^{\pm}(\mathsf{B}) \ar[d,"p_\mathsf{B}^\pm"] \\
D^\pm(\mathsf{A}) \ar[r,dotted,"LF"',"RF"] & D^\pm(\mathsf{B})
\end{tikzcd}\]
\end{definition}

\begin{remark}
In this situation $LF$ / $RF$ is also both the left and right Kan-extension of $Q_\mathsf{B}\circ K(F)$ along the localisation $p_\mathsf{A}$. Concretely: $LF$ fits into a diagram
\[\begin{tikzcd}[sep = large]
K^-(\mathsf{A}) \ar[rr,"p_\mathsf{B}^-\circ K(F)"{name=top_arrow}] \ar[dr,"p_\mathsf{A}^\pm"] && D^-(\mathsf{B}) \\
 & D^-(\mathsf{A}) \ar[ur,bend right=90,"G"{name=G, description}]
\arrow[to=top_arrow,from=G, Rightarrow,"f" description]
\end{tikzcd}
=
\begin{tikzcd}[sep = large]
K^-(\mathsf{A}) \ar[rr,"p_\mathsf{B}^\pm\circ K(F)"{name=top_arrow}] \ar[dr,"p_\mathsf{A}^-"] && D^-(\mathsf{B}) \\
 & D^-(\mathsf{A}) \ar[ur,dotted,"LF"{name=LF, description}] \ar[ur,bend right=90,"G"{name=G,description}] \ar[to=top_arrow,Rightarrow,"\isomorphic" description]
\arrow[Rightarrow,dotted, to=LF, from=G,"\exists!"]
\end{tikzcd}\]
which is commutative up to natural isomorphism $p_\mathsf{B} \circ K(F) \xrightarrow[\isomorphic]{} LF \circ p_\mathsf{A}$ such that for every other functor $D^-(\mathsf{A}) \xrightarrow{G} D^-(\mathsf{B})$ every natural transformation $G \circ p_\mathsf{A} \xrightarrow[\isomorphic]{f} p_\mathsf{B} \circ K(F)$ factors uniquely through this iso.

Therefore some authors \emph{define} $LF$ of \emph{any} additive functor $F$ as the right Kan extension $Ran_{p_\mathsf{A}^-}( p_\mathsf{B}^- \circ K(F))$ and $RF$ as the left Kan extension $Lan_{p_\mathsf{A}^+}(p_\mathsf{B}^+\circ K(F))$. In this situation however, $LF$ and $RF$ do in general not extend $F$.
\end{remark}


\begin{theorem}[Total derived functors exist]
Total derived functors of right / left exact functors exist if $\mathsf{A}$ has enough projectives / injectives.
\end{theorem}
\begin{proof}
Choose a resolution functor $D^\pm(\mathsf{A}) \to K^\pm(\mathsf{A})$ and pre-compose with $p_\mathsf{B}^\pm\circ K^\pm(F)$.
\end{proof}

\begin{remark}
Note that we do not need projective / injective resolutions, $F$-acyclic resolutions are fine too because we have already proven that $F(P_\ast(A))$ is quasi-isomorphic to $F(Q_\ast)$ if $Q_\ast$ is any resolution by $F$-acyclic objects.
\end{remark}

\end{document}