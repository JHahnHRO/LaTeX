% !TeX root = adgc.tex
% !TeX spellcheck = en_GB
\documentclass[fontsize=11pt,fleqn,a4paper]{scrartcl}
\input{_preamble/language_en.tex}
\input{_preamble/math_general.tex}
%%% Own symbols and operators
\newcommand{\IN}{\mathbb{N}}
\newcommand{\IZ}{\mathbb{Z}}
\newcommand{\IQ}{\mathbb{Q}}
\newcommand{\IR}{\mathbb{R}}
\newcommand{\IC}{\mathbb{C}}
\newcommand{\IK}{\mathbb{K}}
\newcommand{\IF}{\mathbb{F}}

\DeclarePairedDelimiter{\abs}{\lvert}{\rvert}
\DeclarePairedDelimiter{\norm}{\lVert}{\rVert}
\DeclarePairedDelimiter{\ceil}{\lceil}{\rceil}
\DeclarePairedDelimiter{\floor}{\lfloor}{\rfloor}

\newcommand{\isomorphic}{\cong}
\newcommand{\homotopic}{\simeq}

\renewcommand{\Im}{\operatorname{\mathfrak{Im}}}
\renewcommand{\Re}{\operatorname{\mathfrak{Re}}}

\DeclareMathOperator{\colim}{colim}

\DeclareMathOperator{\id}{id}
\DeclareMathOperator{\Hom}{Hom}
\DeclareMathOperator{\End}{End}
\DeclareMathOperator{\Irr}{Irr}
\DeclareMathOperator{\IBr}{IBr}

\DeclareMathOperator{\Ind}{Ind}
\DeclareMathOperator{\Res}{Res}

\DeclareMathOperator{\tr}{tr}
\DeclareMathOperator{\sgn}{sgn}
\DeclareMathOperator{\diag}{diag}
\DeclareMathOperator{\ord}{ord}

\DeclareMathOperator{\CharFld}{char}
\DeclareMathOperator{\QuotFld}{Quot}

\DeclareMathOperator{\im}{im}
\DeclareMathOperator{\rad}{rad}

\input{_preamble/math_theorems.tex}
\input{_preamble/math_styles.tex}
\input{_preamble/tikz.tex}
%%%% Styles for algorithms

\usepackage{listings}
\lstset{%
	basicstyle = \ttfamily\small,
	tabsize = 3
}
% Use theorem environment for algorithm descriptions
\newtheoremstyle{algorithms} % Name
			{\bigskipamount}    % Space above
			{\bigskipamount}    % Space below
			{\nopagebreak}      % Body font, also suppress pagebreak between "Theorem 3.14:" and text
			{}                  % Indent amount
			{\bfseries}         % Theorem head font
			{:}                 % Punctuation after theorem head
			{\newline}          % Space after theorem head
			{}                  % Theorem head spec (can be left empty, meaning 'normal')
\theoremstyle{algorithms}
\swapnumbers
\newtheorem{algorithm}{Algorithmus}
% Change numbering of algorithms to include chapter
\renewcommand{\thealgorithm}{A\arabic{algorithm}}
%%%% Tables and figures
\numberwithin{table}{section}
\numberwithin{figure}{section}



\input{_preamble/text.tex}
%\input{_preamble/bibtex_only.tex}
%\input{_preamble/biblatex_bibtex.tex}
%\input{_preamble/biblatex_biber.tex}

% !! Hyperref before imakeidx !!
\input{_preamble/hyperref.tex}
%\input{_preamble/indicies.tex}



\author{Johannes Hahn}
\title{TITLE}
%\subtitle{}

\hypersetup{
pdfinfo=
	{  
		Title={TITLE},
		Author={Johannes Hahn},
		Keywords={KEYWORDS},
		Subject={SUBJECT}
	}
}


\title{Self-injective serial algebras II}

\begin{document}

\maketitle

\begin{convention}
Let $K$ be a field of characteristic $p\in\IP\cup\set{0}$ and $A$ be a finite-dimensional $K$-algebra.
\end{convention}

\begin{lemmadef}[Lemma 11.3.1]
Let $M$ be a finite-dimensional $A$-module. TFAE:
\begin{enumerate}
\item $M$ is \udot{uniserial}, i.e. it has exactly one composition series.
\item $M > J(A)M > J(A)^2M > \ldots > 0$ is a composition series.
\item Every submodule of $M$ is of the form $J(A)^sM$ for some $s\in\IN$.
\item The submodules of $M$ are totally ordered.
\item $M^\ast:=\Hom_K(M,K)$ is uniserial.
\end{enumerate}
In particular: If $M$ is uniserial, then all submodules and all quotients of $M$ are uniserial.
\end{lemmadef}

\begin{remark}
In particular the radical and socle series of $M$ coincide:
\[rad^k(M) = J(A)^kM = soc^{l-k}(M)\] 
where $l=l(M)$ is the length of $M$.
\end{remark}

\begin{definition}
$A$ is called \udot{serial} if all of it finite-dimensional, indecomposable modules are uniserial.
\end{definition}

\begin{theorem}[11.3.4]
Let $A$ be a finite-dimensional, non-simple, serial, indecomposable, self-injective $K$-algebra.

Furthermore let $S_1, \ldots, S_n$ be a full set of representatives of isomorphism classes of simple $A$-modules and let $P_1,\ldots, P_n$ be the corresponding projective covers.

Then:
\begin{enumerate}
\item There is a unique $n$-cycle $\pi\in Sym(n)$ such that
\[J(A)P_i / J(A)^2P_i \isomorphic S_{\pi(i)}\]
\item The $P_i$ all have the same composition length $q$ and the composition factors in
\[P_i > J(A)P_i > J(A)^2P_i > \ldots > J(A)^{q-1}P_i > 0\]
are (in this order) $S_i, S_{\pi(i)}, S_{\pi^2(i)}, \ldots, S_{\pi^{q-1}(i)}$.
\item For all $i$, there is a short exact sequence
\[0 \to S_{\pi^q(i)} \to P_{\pi(i)} \to J(A)P_i \to 0\]
\item If $A$ is symmetric, then $n \mid q-1$.
\end{enumerate}
\end{theorem}

\begin{remark}
Non-simple + indecomposable implies that all the $P_i$ have length $>1$: If $l(P_i)=1$, then $P_i=S_i$ is simple and lies in a single-element block so that $A$ must necessarily be equal to that block because it is indecomposable. In particular there is only one simple module, namely $P_i$ and ${_A A}$ is a sum of copies of $P_i$ and thus semisimple. Being indecomposable, it must be simple by Wedderburn's theorem.
\end{remark}



\end{document}