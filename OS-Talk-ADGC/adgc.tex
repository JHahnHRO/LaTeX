% !TeX root = adgc.tex
% !TeX spellcheck = en_GB
\documentclass[fontsize=11pt,fleqn,a4paper]{scrartcl}
\input{_preamble/language_en.tex}
\input{_preamble/math_general.tex}
%%% Own symbols and operators
\newcommand{\IN}{\mathbb{N}}
\newcommand{\IZ}{\mathbb{Z}}
\newcommand{\IQ}{\mathbb{Q}}
\newcommand{\IR}{\mathbb{R}}
\newcommand{\IC}{\mathbb{C}}
\newcommand{\IK}{\mathbb{K}}
\newcommand{\IF}{\mathbb{F}}

\DeclarePairedDelimiter{\abs}{\lvert}{\rvert}
\DeclarePairedDelimiter{\norm}{\lVert}{\rVert}
\DeclarePairedDelimiter{\ceil}{\lceil}{\rceil}
\DeclarePairedDelimiter{\floor}{\lfloor}{\rfloor}

\newcommand{\isomorphic}{\cong}
\newcommand{\homotopic}{\simeq}

\renewcommand{\Im}{\operatorname{\mathfrak{Im}}}
\renewcommand{\Re}{\operatorname{\mathfrak{Re}}}

\DeclareMathOperator{\colim}{colim}

\DeclareMathOperator{\id}{id}
\DeclareMathOperator{\Hom}{Hom}
\DeclareMathOperator{\End}{End}
\DeclareMathOperator{\Irr}{Irr}
\DeclareMathOperator{\IBr}{IBr}

\DeclareMathOperator{\Ind}{Ind}
\DeclareMathOperator{\Res}{Res}

\DeclareMathOperator{\tr}{tr}
\DeclareMathOperator{\sgn}{sgn}
\DeclareMathOperator{\diag}{diag}
\DeclareMathOperator{\ord}{ord}

\DeclareMathOperator{\CharFld}{char}
\DeclareMathOperator{\QuotFld}{Quot}

\DeclareMathOperator{\im}{im}
\DeclareMathOperator{\rad}{rad}

\input{_preamble/math_theorems.tex}
\input{_preamble/math_styles.tex}
\input{_preamble/tikz.tex}
%%%% Styles for algorithms

\usepackage{listings}
\lstset{%
	basicstyle = \ttfamily\small,
	tabsize = 3
}
% Use theorem environment for algorithm descriptions
\newtheoremstyle{algorithms} % Name
			{\bigskipamount}    % Space above
			{\bigskipamount}    % Space below
			{\nopagebreak}      % Body font, also suppress pagebreak between "Theorem 3.14:" and text
			{}                  % Indent amount
			{\bfseries}         % Theorem head font
			{:}                 % Punctuation after theorem head
			{\newline}          % Space after theorem head
			{}                  % Theorem head spec (can be left empty, meaning 'normal')
\theoremstyle{algorithms}
\swapnumbers
\newtheorem{algorithm}{Algorithmus}
% Change numbering of algorithms to include chapter
\renewcommand{\thealgorithm}{A\arabic{algorithm}}
%%%% Tables and figures
\numberwithin{table}{section}
\numberwithin{figure}{section}



\input{_preamble/text.tex}
%\input{_preamble/bibtex_only.tex}
%\input{_preamble/biblatex_bibtex.tex}
%\input{_preamble/biblatex_biber.tex}

% !! Hyperref before imakeidx !!
\input{_preamble/hyperref.tex}
%\input{_preamble/indicies.tex}



\author{Johannes Hahn}
\title{TITLE}
%\subtitle{}

\hypersetup{
pdfinfo=
	{  
		Title={TITLE},
		Author={Johannes Hahn},
		Keywords={KEYWORDS},
		Subject={SUBJECT}
	}
}


\title{Broué's conjecture in a special case}

\begin{document}

\maketitle

\begin{convention}
Let $G$ be a finite group, $p$ a prime, $(\IK,\mathcal{O},\IF)$ a $p$-modular system with $\IK$ and $\IF$ large enough (e.g. algebraically closed).
\end{convention}

\section{What was Broué's conjecture again?}

\begin{definition}[Derived category]
$D^?(\mathsf{A}) := Q^{-1} K^?(\mathsf{A})$ (with $?\in\set{\text{unbounded},+,-,b}$ where $Q$ is the class of quasi-isomorphisms, i.e.
\[Q:=\Set{f\in \operatorname{Mor}(K) | H(f) \,\text{isomorphism}}\]
\end{definition}

\begin{conjecture}[Abelian Defect Group Conjecture (Broué, Rickard)]
Let $G$ be a finite group, $B\in Bl(G)$ a $p$-block of $G$, $D\leq G$ its defect group and $b\in Bl(N_G(D))$ the Brauer correspondent of $B$.

If $D$ is abelian, then
\[D^b(B) \isomorphic D^b(b)\]
as triangulated categories.
\end{conjecture}

\section{Modular representation theory of $A_5$}

\begin{theorem}[Ordinary character table of $A_5$]
The character table of $A_5$ in characteristic zero is
\[\begin{array}{c|ccccc}
C & 1 & (12)(34) & (123) & (12345) & (13524) \\
\hline
ord(x) & 1 & 2 & 3 & 5 & 5 \\
C_G(x) & A_5 & C_2\times C_2 & C_3 & C_5 & C_5 \\
\abs{C} & 1 & 15 & 20 & 12 & 12 \\
\hline\hline
\chi_1 & 1 & 1 & 1 & 1 & 1\\
\chi_2 & 3 & -1 & 0 & \alpha & \overline{\alpha} \\
\chi_3 & 3 & -1 & 0 & \overline{\alpha} & \alpha \\
\chi_4 & 4 & 0 & 1 & -1 & -1 \\
\chi_5 & 5 & 1 & -1 & 0 & 0
\end{array}\]
with $\alpha:=\frac{1+\sqrt{5}}{2}$.
\end{theorem}

\begin{theorem}[$2$-modular representation theory of $A_5$]
The $2$-modular Brauer character table of $A_5$ is
\[\begin{array}{c|cccc}
 & 1 & (123) & (12345) & (13524) \\
 \hline\hline
\phi_1 & 1 & 1 & 1 & 1 \\
\phi_2 & 2 & -1 & \alpha-1 & \overline{\alpha}-1 \\
\phi_3 & 2 & -1 & \overline{\alpha}-1 & \alpha-1 \\
\phi_4 & 4 & -2 & -1 & -1 \\
\end{array}
\quad
D=\begin{pmatrix}
1 & 1 & 1 & . & 1 \\
. & 1 & . & . & 1 \\
. & . & 1 & . & 1 \\
. & . & . & 1 & .
\end{pmatrix}
\quad
C=\begin{pmatrix}
4 & 2 & 2 & . \\
2 & 2 & 1 & . \\
2 & 1 & 2 & . \\
. & . & . & 1
\end{pmatrix}\]
In particular there are two $2$-blocks:
\begin{itemize}
\item The principal block:
$\IBr(B_0)=\set{\phi_1,\phi_2,\phi_3}$,
$\Irr(B_0)=\set{\chi_1,\chi_2,\chi_3,\chi_5}$,
$D=C_2\times C_2$, $N_G(D) = A_4=D\rtimes C_3$.
\item One block of defect zero:
$\IBr(B_1)=\set{\phi_4}$,
$\Irr(B_1)=\set{\chi_4}$
\end{itemize}
\end{theorem}

\begin{theorem}[$3$-modular representation theory of $A_5$]
The $3$-modular character table of $A_5$ is
\[\begin{array}{c|cccc}
C & 1 & (12)(34) & (12345) & (13524) \\
 \hline\hline
\phi_1 & 1 & 1 & 1 & 1 \\
\phi_2 & 3 & -1 & \alpha & \overline{\alpha} \\
\phi_3 & 3 & -1 & \overline{\alpha} & \alpha \\
\phi_4 & 4 & 1 & -1 & -1 \\
\end{array}
\quad
D=\begin{pmatrix}
1 & . & . & . & 1 \\
. & 1 & . & . & . \\
. & . & 1 & . & . \\
. & . & . & 1 & 1
\end{pmatrix}
\quad
C=\begin{pmatrix}
2 &   &   & 1 \\
  & 1 &   &   \\
  &   & 1 &   \\
1 &   &   & 2  
\end{pmatrix}\]
In particular there are three $3$-blocks:
\begin{itemize}
\item The principal block:
$\IBr(B_0)=\set{\phi_1,\phi_4}$,
$\Irr(B_0)=\set{\chi_1,\chi_4,\chi_5}$,
$D=C_3$, $N_G(D) = C_3 \rtimes C_2$.
\item Two blocks of defect zero:
$\IBr(B_1)=\set{\phi_2}$,
$\Irr(B_1)=\set{\chi_2}$,
$\IBr(B_2)=\set{\phi_3}$,
$\Irr(B_2)=\set{\phi_3}$
\end{itemize}
\end{theorem}

\begin{theorem}[5-modular representation theory of $A_5$]
The $5$-modular character table of $A_5$ is
\[\begin{array}{c|cccc}
C & 1 & (12)(34) & (123) \\
 \hline\hline
\phi_1 & 1 & 1 & 1 \\
\phi_2 & 3 & -1 & 0 \\
\phi_3 & 5 & 1 & -1 \\
\end{array}
\quad
D=\begin{pmatrix}
1 & . & . & 1 & . \\
. & 1 & 1 & 1 & . \\
. & . & . & . & 1 \\
\end{pmatrix}
\quad
C=\begin{pmatrix}
2 & 1 &   \\
1 & 3 &   \\
  &   & 1
\end{pmatrix}\]
In particular there are two $3$-blocks:
\begin{itemize}
\item The principal block:
$\IBr(B_0)=\set{\phi_1,\phi_2}$,
$\Irr(B_0)=\set{\chi_1,\chi_2,\chi_3,\chi_4}$,
$D=C_5$, $N_G(D) = C_5\rtimes C_2$.
\item One block of defect zero:
$\IBr(B_1)=\set{\phi_5}$,
$\Irr(B_1)=\set{\chi_5}$
\end{itemize}
\end{theorem}

\section{Step minus one: Defect zero}

\begin{remark}
ADGC is trivially true for defect zero, because $N_G(1) = G$ and $B=b$ in this case.
\end{remark}

\begin{theorem}
All blocks of defect zero are matrix rings.
\end{theorem}
\begin{proof}
Standard theorem shows that blocks of $\IF G$ of defect zero are simply matrix rings $\IF^{a\times a}$.

Sketch: Defect zero $\overset{V\leq D}{\implies}$ All vertices trivial $\implies$ all simple modules of this block are projective $\implies$ all modules are projective $\implies B$ is semisimple $\overset{Wedderburn}{\implies} B \isomorphic \IF^{\dim(S)\times\dim(S)}$ because $B$ is indecomposable.
\end{proof}

\section{Step 0: Different equivalences}

\begin{theorem}[Morita]
Let $A$ and $B$ be two $k$-algebras. TFAE:
\begin{enumerate}
\item $A\mathsf{-Mod} \isomorphic B\mathsf{-Mod}$.
\item $A\mathsf{-proj} \isomorphic B\mathsf{-proj}$.
\item There ex. bimodules ${}_A M_B$ and ${}_B N_A$ s.t.
\[{}_A M_B \otimes {}_B N_A \isomorphic {}_A A_A \quad\text{and}\quad {}_B N_A\otimes {}_A M_B \isomorphic {}_B B_B\]
In this case $M$ and $N$ determine each other via $N\isomorphic\Hom_{\mathsf{Mod-}B}(M,B)$ and $M\isomorphic\Hom_{\mathsf{Mod-}A}(N,A)$. Moreover $A\isomorphic\End_{\mathsf{Mod-}B}(M)$ and $B\isomorphic\End_{\mathsf{Mod-}A}(N)$.
\item There ex. a \udot{progenerator}, i.e. a module $M$ s.t.
\begin{enumerate}
\item $M$ is f.g. projective and
\item $M$ is a generators of $A\mathsf{-Mod}$, i.e. every module $X$ is a quotient of $\bigoplus_{i\in I} P$ for some sufficiently large $I$.
\end{enumerate}
which satisfies $B\isomorphic\End(M)$.
\end{enumerate}
\end{theorem}
\begin{proof}
a.$\implies$b. because finite generation and projectivity can be categorically defined.

b.$\implies$c. If an equivalence $A-\mathsf{proj} \xtofrom[G]{F} B-\mathsf{proj}$ is given, then $M:=G({}_A A)$ and $N:=F({}_B B)$ are the desired bimodules. In fact $F=N\otimes-$ and $G=M\otimes-$ because $F$ and $G$ are additive.

c.$\implies$a. Conversely if $M,N$ are given, then $F:=N\otimes-$ and $G:=M\otimes-$ are pseudo-inverse functors.
\end{proof}

\begin{remark}
$\Hom(M_B,B_B)$ is isomorphic to $M^\vee=\Hom_k(M,k)$ if $B$ is a symmetric $k$-algebra. Similarly $\Hom(N_A,A_A)\isomorphic N^\vee$ if $A$ is symmetric.
\end{remark}

\begin{theorem}[Rickard, Keller, ...]
Let $A$ and $B$ be two $k$-algebras which are f.g. projective over $k$. TFAE:
\begin{enumerate}
\item $D^b(A) \isomorphic D^b(B)$ as triangulated categories.
\item $K^b(A\mathsf{-proj}) \isomorphic K^b(B\mathsf{-proj})$ as triangulated categories.
\item There exist $P$ of $A$-$B$-bimodules and $Q$ of $B$-$A$-bimodules s.t.
\[P \otimes_B^L Q \sim A \quad\text{and}\quad Q \otimes_A^L P \sim B\]
In this case $P$ and $Q$ determine each other via $Q=\Hom_{D^b(\mathsf{Mod-}B)}(P,B)$ and $P=\Hom_{D^b(\mathsf{Mod-}A)}(Q,A)$. Moreover $A\isomorphic\End_{D^b(B)}(P)$ and $B\isomorphic\End_{D^b(A)}(Q)$.
\item There exists a \udot{tilting complex}, i.e. a bounded complex $P$ of $A$-modules s.t.
\begin{enumerate}
\item $P$ consists of f.g. projective $A$-modules
\item $\operatorname{add}(P)$, the smallest full subcategory which contains $P$ and is closed under taking direct sums and direct summands, generates $K^b(A-\mathsf{proj})$ as a triangulated category.
\item $P$ is \enquote{rigid}:
\[\forall n\neq 0: \Hom(P,P[n]) = 0\]
\end{enumerate}
such that $B\isomorphic\End({}_A P)$
\end{enumerate}
\end{theorem}

\begin{remark}
Now it is not (known to be) true that every equivalence $F: D^b(A)\to D^b(B)$ is actually isomorphic to some $Q\otimes^L-$ as it is in the Morita case. If so, $F$ is called \enquote{standard}.
\end{remark}

\begin{definition}[Rickard]
Let $G,H$ be two finite groups with a common (fixed) $p$-subgroup $D$.

A \udot{splendid} tilting complex for two blocks $B\in Bl(\IF G)$ and $C\in Bl(\IF H)$ is a complex $B$-$b$-bimodules as above such that additionally the following hold:
\begin{enumerate}
\item Homotopy instead of quasi-isomorphism:
\[P \otimes_b P^\vee \homotopic B \quad\text{and}\quad P^\vee \otimes_B P \homotopic b\]
\item Each term of $P$ is a $p$-permutation module of $G\times H$ and is projective relative to $\Delta(D)$.
\end{enumerate}
\end{definition}

\begin{theorem}[Rickard]
Let $A$ be a self-injective $\IF$-algebra. The canonical functor $A\mathsf{-mod} \to D^b(A)$ which maps a module $M$ to the complex $\cdots\to0\to M\to 0\to\cdots$ with $M$ concentrated in $0$ induces an equivalence
\[\underbrace{A\mathsf{-mod}/A\mathsf{-proj}}_{=A\mathsf{-\underline{mod}}} \to D^b(A) / K^b(A\mathsf{-proj})\]
of triangulated categories.
\end{theorem}

\begin{corollary}
Let $A$ and $B$ be finite-dimensional, self-injective $\IF$-algebras. If they are derived equivalent, they are also stably equivalent. In fact, there is a stable equivalence of Morita type.
\end{corollary}

\begin{theorem}[Okuyama, Rickard, ...]
Let $A$ and $B$ be symmetric $k$-algebras over a field.

If $\mathcal{F}: A\mathsf{-mod}\to B\mathsf{-mod}$ is an exact functor which induces a stable equivalence, $\Irr(A)=\set{S_1,\ldots,S_n}$ are and $\mathcal{X}=\set{X_1,\ldots,X_n}$ are objects in $D^b(B)$ s.t.
\begin{enumerate}
\item $\Hom(X_i,X_j[m]) = 0$ for all $m<0$ and all $i,j$.
\item $\Hom(X_i,X_j) = \begin{cases} k & i=j \\ 0 & i\neq j\end{cases}$
\item $\mathcal{X}$ generates $D^b(B)$ as triangulated category.
\end{enumerate}
and such that $X_i$ is stably isomorphic to $\mathcal{F}(S_i)$ for all $i$ (i.e. isomorphic in $D^b(A)/K^b(A\mathsf{-proj})$), then $\mathcal{F}$ also induces a derived equivalence.
\end{theorem}

\begin{remark}
The proof is based on a theorem by Linckelmann that a stable equivalence of Morita type (i.e. induced by a tensor-functor) between indecomposable, finite-dimensional, self-injective $K$-algebras which also maps simples to simples is a Morita equivalence.
\end{remark}

\section{Step one: Cyclic defect}

\begin{theorem}
Blocks with cyclic defect group $D$ are Brauer tree algebras with $e=\abs{\IBr(B)}$ edges and multiplicity $\mu = \frac{\abs{D}-1}{e}$.
\end{theorem}

\begin{theorem}[Rickard]
ADGC holds for blocks of cyclic defect. In fact, all Brauer tree algebras with the same number of edges and the same multiplicity are derived equivalent.
\end{theorem}

\section{Step two: Klein four defect}

\begin{theorem}[$2$-modular representation theory of $A_4$]
The $2$-modular Brauer character table of $A_4$ is
\[\begin{array}{c|ccc}
 & 1 & (123) & (132) \\
\hline\hline
\psi_1 & 1 & 1 & 1 \\
\psi_2 & 1 & \zeta_3 & \zeta_3^2 \\
\psi_3 & 1 & \zeta_3^2 & \zeta_3
\end{array} \quad
D=\begin{pmatrix}
1 & . & . & 1 \\
. & 1 & . & 1 \\
. & . & 1 & 1
\end{pmatrix}
\quad
C=\begin{pmatrix}
2 & 1 & 1 \\
1 & 2 & 1 \\
1 & 1 & 2 
\end{pmatrix}\]
In particular there is only one $2$-block.
\end{theorem}

\begin{example}
ADGC holds for $A_5$.
\end{example}
\begin{proof}
$p=3$ and $p=5$ are already done because cyclic defect.

Only other case is the principal $2$-block. We set $D:=V_4\in Syl_2(G)$, $H:=N_G(D) = A_4$.

Since $D$ is a TI subgroup of $A_5$, Green correspondence gives a stable equivalence of Morita type
\[\Set{M\in\IF H\mathsf{-\underline{mod}} | vx(M)\leq D} \xtofrom[\Res_H^G]{\Ind_H^G} \Set{M\in\IF G\mathsf{-\underline{mod}} | vx(M)\leq D}\]
which restricts to a stable equivalence of Morita type
\[b\mathsf{-\underline{mod}} \xtofrom[\Res_H^G]{\Ind_H^G} B\mathsf{-\underline{mod}}\]

The restriction of the three simple $B$-modules $\phi_1,\phi_2,\phi_3$ are
\[Y_1 := \psi_1 \quad Y_2 = \begin{array}{c}\psi_2\\\psi_3\end{array} \quad Y_3=\begin{array}{c}\psi_3\\\psi_2\end{array}\]
which can be verified by explicit calculations. Simple constituents can be seen from the character tables. Which one is the socle and which the head of $Y_i$ can be calculated by looking at explicit modules.

Then $X_1:=Y_1$ is already simple. Furthermore
\[\Omega Y_2 = \begin{array}{c} \psi_1 \\ \psi_2 \end{array} \quad \Omega Y_3 = \begin{array}{c} \psi_1 \\ \psi_3 \end{array}\]
Then we can set $X_2:=\Omega Y_2[1]$ and $X_3:=\Omega Y_3[1]$. These three generate $D^b(b)$ because $\psi_1=X_1$ is already in $\mathcal{X}$, $\psi_2$ and $\psi_3$ are kernels of $X_i \twoheadrightarrow X_1$. It is also easy to see that $\dim_k\Hom(X_i,X_j) = \delta_{ij}$ and that $\Hom(X_i,X_j[m])=0$ for $m<0$.

Okuyama's method therefore upgrades the stable equivalence to a derived equivalence.
\end{proof}

\begin{remark}
In fact, one can do the same with the cyclic defect groups instead of using the heavy guns.
\end{remark}


\end{document}